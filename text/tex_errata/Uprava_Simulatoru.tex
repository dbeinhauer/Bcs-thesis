\chapter{Úpravy v programu}

\section{Oprava načítání mapy}
V programu z adresáře \texttt{map\_preprocessing} jsme opravili chybu v převodu
jednotek u délek jednotlivých silnic. A vytvořili opravené vstupní soubory simulátoru 
z adresáře \texttt{prepared\_map}. V opravené reprezentaci jsou již v souborech 
\texttt{edges.txt} a \texttt{prepared\_edges.txt} uloženy vzdálenosti v kilometrech.

Dále jsme při extrakci mapy z OpenStreetMaps pomocí nástroje Osmium zahrnuli do 
výběru silnic, také silnice reprezentující nájezdy a výjezdy z dálnic, silnic pro
motorová vozidla a silnic 1. třídy (přidáním silnic typu, jehož postfix je \texttt{\_link}).
Opravená mapa již obsahuje sjezdy a nájezdy na zvolené typy silnic.
Takto upravená silniční síť má po odstranění vrcholů stupně 2, celkem $8477$ vrcholů a 
$10906$ hran. Mapa se rozpadá na 22 komponent souvislosti. Rozpad je způsoben malými úseky 
silnic v pohraničních oblastech, které byly vlivem extrakce mapy odpojeny od silniční sítě.
S malými komponentami souvislosti nakládáme stejně jako v původní variantě 
(napojíme je k nejbližšímu vrcholu).

\section{Úpravy v simulátoru}
\label{sec:upravy}
V simulátoru jsme z důvodu lepší výpočetní efektivity změnili způsob výběru nabíjecí stanice.
V nové variantě si před začátkem simulace předpočítáme nejbližší nabíjecí stanici pro
každou křižovatku v silniční síti. V momentě, kdy se vozidlo rozhodne jet na nabíjecí stanici,
přeplánuje trasu tak, aby mířilo na předpočítanou nejbližší nabíjecí stanici od křižovatky, 
v níž se aktuálně nachází. Původní variantu lze zprovoznit odkomentováním označených úseků
v souboru \texttt{Map.cpp}.

Dále jsme v optimalizaci pomocí k-means zjednodušili výpočet vah křižovatek. V nové 
implementaci je váha křižovatky rovna součtu průměrných hladin baterie potřebných pro nabití vozidla
na přilehlých silnicích. Změnu jsme provedli, neboť jsme při spuštění optimalizační metody
narazili na přetékání vah z důvodu velmi malých hodnot.

Navíc jsme řádně přenastavili základní nastavení, které je nyní nastaveno na nastavení
simulátoru používané v experimentech.

Také jsme narazili na chybu ve výpisu doby trvání jedné trasy, proto v optimalizaci naprosto
vynecháváme tento parametr (ve výpočtu loss funkce).


\subsection{Informace o simulaci}
Do simulátoru jsme navíc přidali výpis dalších informací pro podrobnější popis výsledků simulace. 
Mezi přidané informace patří informace vytížení jednotlivých nabíjecích stanic (počet zákazníků
a průměrná doba čekání vozidel ve frontě na stanici). 

Dále jsme přidali výpis počtu vozidel, které ukončily cestu. Dojezd do cílové destinace a návrat 
z cílové destinace do počátku považujeme
za 2 různé cesty. Každé vozidlo by tedy v ideálním případě mělo přispět do tohoto počtu dvakrát.
Vyjímkou jsou vozidla, které se vybily, či nestihly dojet do cíle před skončením simulace.
Tyto vozidla do tohoto počtu nepřispívají. Na konci simulace by tedy tento počet měl být roven 
přibližně dvojnásobku všech vygenerovaných vozidel.

Mezi další přidané informace patří počet vozidel, které nestihly dojet do cíle před 
doběhnutím simulace, počet vozidel, jež se vybily během cesty na nabíjecí stanici, a 
celkový počet vozidel, které nedorazily do cílové destinace (vybily se nebo skončila simulace).

Nakonec jsme dodali také podrobnější přehled o všech nabíjecích stanicích. Přidali jsme
informaci o počtu stanic, které byly aspoň jednou během simulace využity a výpis počtu zákazníků 
a průměrné doby nabíjení na jednotlivých stanicích.


