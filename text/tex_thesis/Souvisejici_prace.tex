\chapter{Související práce}
\label{chap:souvisejici_prace}

V této kapitole si popíšeme několik souvisejících prací a stručně popíšeme 
jejich vlastnosti.


Simulátor SUMO\footnote{\url{https://www.eclipse.org/sumo/}} je velmi detailní
simulátor dopravy v rozsahu funkcionalit zdaleka přesahující naši implementaci.
V programu lze do jisté míry simulovat také elektromobily. Pro naše učely je
ale tento program velmi komplexní a práce s optimalizačními algoritmy
by tak byla neprakticky komplikovaná.

Simulátor City Flow\footnote{\url{https://cityflow.readthedocs.io/en/latest/index.html}}
se pyšní především výpočetní efektivitou\footnote{\url{https://cityflow.readthedocs.io/en/latest/introduction.html}}.
Bohužel nenabízí jednoduché uživatelské rozhraní pro rozšíření o metody
potřebné pro naši optimalizaci.

V práci \citet{kmeans_layout} je navržen simulátor dopravy pro následnou 
analýzu navě navrženého optmalizačního algoritmu rozmístění nabíjecích stanic
na území Japonska. Optimalizační algoritmus v práci se snaží minimalizovat 
počet vozidel, kterým se vybila baterie, pomocí posunů nabíjecích stanic směrem
k místům, kde se v minulosti vybila baterie vozidla. Návrh simulátoru naší práce
a optimalizační algoritmus s použití metody K-Means jsou inspirovány zmíněnou
prací.

V práci \citet{niccolai2021optimization} je zkoumáno několik variant evolučních
algoritmů pro optimalizaci rozmístění nabíjecích stanic elektrických vozidel
ve městě Milán. V práci byl porovnáván také hladový přístup rozmísťující 
iterativně nabíjecí stanice na místa lokálních extrémů specificky zadefinované
ztrátové funkce. 

Práce \citet{zhu2016charging} se zabývá optimalizací rozmístění nabíjecích 
stanic v okolí města Peking užitím technik genetického algoritmu. V práci 
jsou porovnávány 2 varianty modelů popisující možné pozice nabíjecích stanic a
je v ní poukázáno, že správná volba modelu ovlivňuje kvalitu řešení optimalizace. 

V práci \citet{kinay2021full} je navžen simulátor pro rozvrhování pozic 
nabíjecích stanic. Problém je zde matematicky popsán. V práci jsou navrženy
2 varianty optimalizací. Jedna minimalizující celkovou cenu rozmístění 
nabíjecích stanic s minimalizací odchylek od původní trasy, druhá optimalizuje
pozice stanic s ohledem na minimalizaci odchylek.