\chapter{Příprava mapy silniční sítě}
\label{chap:priprava_mapy}

Hlavní motivací pro vytvoření simulátoru dopravy je vytvoření uživatelsky
přívětivého prostředí pro práci s různými variantami rozmístění nabíjecích 
stanic na reálné dopravní síti. Je tedy více než žádoucí pracovat s reprezentací
reálné silniční sítě. K tomuto účelu používáme volně dostupné mapy formátu
\texttt{.osm.pbf} ze serveru 
Geofabric\footnote{\url{https://download.geofabrik.de/}}. 

V této kapitole popisujeme jakým způsobem lze zmíněné mapy upravit do formátu
požadovaného naším simulátorem a poukazujeme na důležité poznatky této problematiky. 
V našich experimentech pracujeme s reprezentací mapy 
České republiky\footnote{\url{https://download.geofabrik.de/europe/czech-republic.html}},
která je přiložena k práci v již požadovaném formátu simulátoru a pro správné spuštění
simulátoru tak postupy popsané v této kapitole nejsou nutné. 


\section{Extrakce dopravní sítě}
\label{sec:extrakce_site}

V této sekci popisujeme proces extrakce dopravní sítě z mapy formátu 
\texttt{.osm.pbf} do formátu požadovaného našim simulátorem a poukazujeme
na specifické aspekty námi zvolené mapy pro analýzu optimalizačních metod.


\subsection{Extrakce s pomocí nástroje Osmium}

Soubory formátu \texttt{.osm.pbf} obsahují mimo informace o dopravní síti,
také velké množství informací irelevantních pro funkci našeho simulátoru. 
Důsledkem této skutečnosti je několikanásobně vyšší množství dat pro 
zpracování, což mimo jiné vede k vyšší paměťové i časové náročnosti programu.
Je tak žádoucí všechny nepotřebné informace odstranit. K tomuto účelu
používáme nástroj \texttt{Osmium}\footnote{\url{https://osmcode.org/osmium-tool/}}.
S jehož pomocí jsme schopni z mapy vyextrahovat silniční síť a zároveň
vybrat pouze námi zvolené druhy silnic.

\subsubsection{Dodatek k výběru druhů silnic}
Mezi zvolené typy silnic, které náš simulátor rozeznává, patří 
dálnice, silnice pro motorová vozidla a silnice první třídy (v kontextu
dat se jedná o typy \texttt{motorway}, \texttt{trunk} a \texttt{primary}).
Silnice nižších tříd jsme zanedbali, neboť předpokládáme, že výše zmíněné 
silnice tvoří souvislý graf, případně, že jsou rozděleny do relativně nízkého 
počtu komponent souvislosti a dostatečně aproximují zkoumanou dopravní síť.
S ohledem na skutečnost, že v reálném případě je silniční síť v naprosté většině
případů souvislý graf, je žádoucí minimalizovat rozpad dopravní sítě na 
komponenty souvislosti. V našem případě je graf rozdělen do 75 komponent 
souvislosti, což se jeví jako poměrně vysoké číslo a vyvolává pochybnosti,
zda je naše volba typů silnic vhodná. Vysoký počet komponent si odůvodňujeme
výběrem mapy pokrývající pouze výřez reálné oblasti, což může vézt k rozpadu
silniční sítě na více komponent souvislosti především v hraničních oblastech.

Významným faktorem podněcujícím k eliminaci silnic nižší třídy je fakt, že
s klesající třídou silnic typicky několikanásobně roste jejich počet. Výsledný
graf je pak mnohonásobně hustší, což vede k vyšší výpočetní náročnosti celého
simulátoru, případně na méně výkonných strojích k nepoužitelnosti simulátoru
z časových i paměťových důvodů. V našem případě vzrostl počet hran grafu při
zahrnutí silnic 2. třídy asi pětinásobně. 

Předpokládáme, že neexistuje mnoho hustě osídleným oblastí v simulované oblasti,
které bychom takto vyloučili ze simulace. Jelikož předpokládáme, že hustěji
osídlené oblasti mají v blízkém okolí přístup k silnicím vyšších tříd. 
Nevýhodou tohoto přístupu je zanedbání realistické reprezentace sítě na úrovni
jednotlivých měst, či malých oblastí. Nemůžeme tak dostatečně realisticky 
simulovat například průjezdy vozidel křižovatkou, dopravní zácpy ve městech apod.
Zároveň jsme omezeni na cesty vozidel vedoucích pouze po silnicích vyšší třídy. 
Předpokládáme však že tato omezení nejsou s ohledem
na námi řešený problém nijak závažná, neboť nás zajímají především dálkové 
cesty, během nichž je častější potřeba řidiče použít nabíjecí stanici.


\subsubsection{Extrakce měst}

Pro dosažení co nejrealističtějších výsledků simulace je potřeba vhodně
generovat počáteční a cílové pozice vozidel. K tomuto účelu nám slouží informace
o pozicích a počtu obyvatel osídlených oblastí na mapě. 
Pro získání těchto informací nám znovu poslouží nástroj \texttt{Osmium}. 
S jeho pomocí získáváme informace o městech (oblasti označené jako \texttt{city}
nebo \texttt{town}). 

Vynechali jsme menší osídlené oblasti, jelikož předpokládáme, že
počet výjezdů a cílů cesty je v této oblastí zanedbatelný. Navíc je počet
menších celků vysoký a jejich zahrnutí by vyžadovalo více výpočetního výkonu.


\subsection{Příprava dat pomocným skriptem}

Postupy popsanými výše jsme získané informace zásadně zredukovali. Veškerá 
vyfiltrovaná data jsou ale stále ve formátu \texttt{osm.pbf}, který musíme 
vhodným způsobem zpracovat a odstranit nepotřebné informace o silnicích
a obytných oblastech. K tomuto účelu jsme využili knihovny programovacího
jazyka Python \texttt{Pyrosm}\footnote{\url{https://pyrosm.readthedocs.io/en/latest/}}.
A naimplementovali program v jazyce Python, jenž načte data ze souborů ve formátu
\texttt{osm.pbf} a náležitě je upraví do podoby vstupních souborů našeho 
simulátoru dopravy. Podrobnější informace k přípravě dat i veškeré programy 
jsou dodány v příloze v adresáři \texttt{map\_processing}. 


\section{Dodatek k volbě programovacího jazyka a nutnosti předpřípravy dat}

Pro správné fungování simulátoru není práce s nástroji popsanými v sekci (\cref{sec:extrakce_site}).
nutná (je například možné pracovat s ručně připravenou reprezentací
dopravní sítě a nerešit jakoukoli přípravu dat z reálných map).

Reprezentace mapy ve formátu \texttt{osm.pbf} je obecně velmi rozšířená, což má za 
následek existenci početné skupiny knihoven (především v jazyce Python), jež umí
s těmito soubory pracovat a vhodným způsobem je zpracovávat. Zároveň existuje 
řada knihoven programovacího jazyka Python, které nabízejí širokou
škálu funkcionalit pro práci s grafy a grafovými algoritmy. 
Může nám tedy připadat kontraproduktivní pracovat s několika různými 
nástroji pro předpřípravu dat a pracovat s více programovacími jazyky, 
když lze většinu funkcionalit pokrýt programem napsaným v jazyce Python s 
využitím vhodných knihoven. 
Původním cílem práce bylo zvolit tento přístup. Během práce na simulátoru 
jsme však narazili na vysokou paměťovou i časovou náročnost tohoto postupu
a upustili jsme od něj.  
Na základě těchto skutečností jsme se nakonec rozhodli v jazyce Python 
pouze předzpracovat data a zbytek simulátoru naimplementovat v 
jazyce C++ s využitím knihovny \texttt{boost}\footnote{\url{https://www.boost.org/}}.
