\chapter{Průběh simulace}
\label{chap:prubeh_simulace}

V této kapitole popisujeme průběh a princip simulace dopravy v simulátoru, jenž
popisuje \cref{chap:popis_simulatoru}.

Pro simulování dopravy v připravené dopravní síti používáme \emph{diskrétní simulaci}.
Jedná se o typ simulace, jenž pracuje v tzv. 
diskrétním čase (skokově). Slouží pro zjednodušení simulace komplexního 
systému pomocí vhodné abstrakce.

Simulace typicky používá parametry jako je čas (mění se skokově a určuje pořadí
událostí v simulaci), údálost (změny v systému), fronta (čekání entit na 
provedení události) a podobně. Příkladem diskrétní simulace může být popis obchodního
střediska, kde simulujeme příchody a odchody zákazníků, jejich čekání ve frontě 
u pokladen atd. Více informací o diskrétní simulaci popisuje \citet{matloff2008introduction}.


\section{Inicializace simulátoru}

Před prvním spuštěním simulátoru je nejprve potřeba řádně načíst a inicializovat 
mapu simulace a veškeré její parametry. \Cref{chap:priprava_mapy} popisuje podrobněji
proces předpřípravy mapy a \cref{chap:popis_simulatoru} definuje podobu simulátoru.

Následně je pak možné opakovaně spouštět simulaci dopravy. Před začátkem každé 
simulace jsou vždy vynulovány veškeré statistické údaje popisující vlastosti 
simulovaného prostoru, například informace o průměrném stavu baterie v jednotlivých
segmentech (z \cref{def:segment_silnice}), o průměrné době čekání na nabíjecích 
stanicích, počtu vozidel, kterým se vybila baterie a podobně. 

Dále je před každým spuštěním simulace nutné rozmístit nabíjecí stanice do dopravní sítě.
Ty mohou být rozmístěny buď náhodně, nebo za pomoci zvolené optimalizační metody.


\section{Popis simulace}

Dopravní síť simulujeme pomocí diskrétní simulace. Pro správnou 
funkcionalitu musí simulátor obsahovat časový plán, jenž chronologicky 
rozvrhuje vykonávání akcí v čase. Především je potřeba definovat všechny
možné události a jejich důsledky v simulovaném prostoru. 
V aktuální implementaci pracujeme pouze s jedním typem vozidla, kterým je typ auto
(viz. \cref{subsec:druhy_vozidel}). Všechny popisované vlastnosti
simulace se tedy vztahují pouze na tento objekt. 


\section{Události simulace}
Pro správný popis diskrétní simulace musíme zadefinovat vhodné události v simulaci.
Události v simulaci jsou dány typem události, časem vykonávání a objektem 
zapojeným do události. S jejich pomocí jsme schopni popsat veškeré jevy v 
simulovaném prostoru.
Veškeré události simulace se vážou na konkrétní vozidlo, neboť se jedná o 
jediný prvek simulace, jenž svým chováním přímo ovlivňuje průběh simulace
v čase.

Zvolili jsme 6 typů událostí, které vhodně popisují simulované prostředí s ohledem
na námi řešený optimalizační problém. Mezi tyto události patří výjezd vozidla,
přesun vozidla po hraně grafu, čekání ve frontě na nabíjení, nabíjení,
konec nabíjení a čekání auta na návrat.

\subsection{Výjezd vozidla}

Událost \emph{výjezd vozidla} (v simulátoru označován jako \texttt{START}) je 
jedna ze stěžejních událostí simulace. Během této události je naplánován 
první pohyb vozidla po silnici v grafu simulátoru (viz. \cref{sec:dopravni_sit})
a zároveň je zde vygenerováno nové vozidlo a naplánován jeho čas výjezdu, 
dle pravidel ze \cref{subsec:generovani_vozidel}.

\subsection{Pohyb vozidla}
\label{subsec:udalost_presunu}

Událost \emph{pohyb vozidla} (\texttt{RUNNING}) po silnici (viz. \cref{def:silnice})
je s přehledem nejčastější událost v simulaci. V této události je simulován 
přejezd vozidla po silnici v grafu dopravní sítě simulátoru.

Tato událost reprezentuje přejezd vozidla po části silnice (nikdy nepřesáhne hranice 
silnice), jenž je spravován samotným vozidlem. S její pomocí je naplánován 
čas dojezdu vozidla na konec vybraného úseku hran grafu. Na tento čas je také
naplánována následující událost vozidla. Dále je zde nasimulován přejezd vozidla
po zvolené posloupnosti hran, vybíjení baterie při přesunu a
je aktualizována informace o hustotě provozu na silnicích.
Tato událost je také jedinou příležitostí, kdy se může vozidlo rozhodnout
změnit svou trasu a vydat se směrem k nejoptimálnější nabíjecí stanici 
(viz. \cref{subsec:zajizdeni_na_nabijecku}).

Důvod volby pohybu po souvislé části silnice místo po jednotlivých hranách grafu
je motivován myšlenkou zjednodušené dopravní sítě 
z \cref{def:zjednodusena_dopravni_site}. Víme totiž, že většina naplánované cesty
je složena ze silnic (viz. \cref{subsec:hledani_trasy}) 
a s ohledem na výpočetní výkon je zbytečně náročné 
simulovat přejezdy po jednotlivých hranách. 
Nevýhodou této implementace je komplikovanější simulace vybíjení baterie a 
nemožnost simulovat hustotou provozu na jednotlivých hranách odděleně, čímž simulace 
ztrácí na realističnosti.


\subsection{Události nabíjení vozidla}
Proces nabíjení vozidla vyžaduje nejvíce událostí pro jeho popsání 
ze všech procesů v simulaci.

První událostí je \emph{čekání vozidla ve frontě} na nabíjení 
(\texttt{WAITING\_LINE}), jež vždy nastane při dojezdu vozidla na nabíjecí
stanici i v případě, že ve stanici je volná nabíječka a není tak potřeba čekat
v řadě. Během této události je vozidlo zařazeno do fronty příslušné nabíjecí 
stanice a čeká zde na dobu přiřazení k nějakému nabíjecímu slotu stanice.

V momentě přiřazení slotu se přesune do události \emph{nabíjení vozidla}
(\texttt{CHARGING}). Ta pokrývá většinu interakce vozidla na 
nabíjecí stanici. Je zde zvoleno na jakou hladinu baterie bude vozidlo nabito.
Dále je zde naplánován čas ukončení nabíjení vozidla a je simulováno 
nabíjení vozidla.

Po uplynutí času potřebného pro nabíjení vozidla následuje 
událost \emph{konce nabíjení} (ozn. \texttt{END\_CHARGING}).
V této události je naplánována nová optimální trasa vozidla, kterému je vrácena
možnost se v budoucnu rozhodnout směřovat k nabíjecí stanici. Je zde také
naplánováno nabíjení dalšího vozidla čekajícího v řadě.

\subsection{Konec cesty vozidla}

Poslední událostí simulace je čekání vozidla na návrat (\texttt{WAITING\_RETURN}).
V této událost vozidlo pouze čeká náhodnou dobu z exponenciální distribuce.
Na dobu ukončení čekání naplánuje následující pohyb vozidla 
(viz. \cref{subsec:druhy_vozidel}). 

Pokud se již vozidlo vracelo zpět do místa výjezdu, pak je vozidlo řádně
odstraněno ze simulace. 

Proces návratu je typický pouze pro druh vozidla s názvem auto 
(viz. \cref{subsec:druhy_vozidel}), protože je ale v naší implementaci
zahrnut pouze jeden druh vozidla, můžeme tuto vlastnost zobecnit na 
všechny vozidla.


\section{Běh simulace}

Průběh simulace je poměrně přímočaný. Při spuštění je zadána doba simulace a 
všechny další potřebné parametry a je vygenerováno první vozidlo simulace.
Následně jsou již pouze chronologicky vykonávány naplánované události do doby
než je překročena doba simulace, která následně končí. Po skončení lze ze simulace
získat relevantní informace pro optimalizaci. Případně lze veškeré informace
vynulovat a spustit novou simulaci.