\chapwithtoc{Závěr}
\label{chap:zaver}

V práci jsme navrhli simulátor dopravy speciálně navržen pro snadnou optimalizaci
rozmístění nabíjecí stanic v dopravní síti. A následně zkoumali několik 
vybraných variant optimalizace na mapě dopravní sítě České republiky.

Z důvodu vysoké výpočetní náročnosti zkoumaného problému jsme byli nuceni
některé aspekty simulace značně zjednodušit, což částečně vedlo k 
nerealističnosti výsledků. Ve značné míře jsme například museli zredukovat 
simulovanou silniční síť a simulování cest vozidel na krátké vzdálenosti.
Dalším ze závažných nedostatků simulace je nerealistický počet vozidel,
kterým se vybije baterie během cesty. Hlavním důvodem tohoto nedostatku
je potřeba vhodně zadefinovat chybovou funkci pro optimalizaci problému
(viz. \cref{sec:loss}). Částečně tuto skutečnost způsobuje také velmi
zjednodušený proces volby nabíjecí stanice vozidlem 
(viz. \cref{subsec:zajizdeni_na_nabijecku}).
Tato skutečnost je částečně způsobena nevhodným algoritmem pro volbu nabíjecí
stanice, jenž musel být podstatně zjednodušen v důsledku potřeby výpočetně
efektivního simulátoru. Na základě experimentů, které podrobně popisuje 
kapitola \cref{chap:analysis}, jsme však došli k názoru, že námi navržený 
simulátor popisuje dostatečně kvalitně aspekty dopravní sítě, jenž jsou 
potřebné pro optimalizaci rozmístění nabíjecích stanic. 

Po provedení experimentů jsme došli k závěru, že hladový optimalizační
algoritmus (viz. \cref{sec:greedy}) a především optimalizace pomocí genetického
algoritmu (viz. \cref{sec:genetic_optim}) vykazují lepší výsledky než náhodný
přístup, založený na rozmísťování stanic podle počtu obyvatel v jednotlivých
částech dopravní sítě (např. v genetickém algoritmu se pro 2000 nabíjecích 
stanic snížil počet vozidel s vybitou baterií v simulaci o $\frac{1}{6}$).
Především u optimalizace genetickým algoritmem, pozorujeme, že kvalita řešení 
by se mohla ještě vzrůst při zvýšením počtu generací a velikosti populace.
Tyto výsledky, jsme však vzhledem k výpočetní náročnosti optimalizace nebyli 
schopni ověřit. Na druhou stranu optimalizační algoritmus inspirovaný algoritmem
k-means (\cref{sec:kmeans_optim}) dosahoval dokonce mírně horších výsledků jako
náhodný přístup. Tuto skutečnost si odůvodňujeme velkou mírou aproximace při 
hledání vhodné nabíjecí stanice a malým počtem iterací algoritmu v experimentech.


\section*{Budoucí práce}

Do budoucna se nabízí doimplementovat do simulátoru vhodnější algoritmus pro 
výběr nabíjecí stanice, čímž by se mohla zlepšit realističnost chování vozidel
při jízdě na nabíjecí stanice a výrazně snížit počet vybitých vozidel v 
simulaci.

Dále se nachází potenciál ve zlepšení optimalizační 
algoritmu pomocí algoritmu k-means, který pravděpodobně dosahuje špatných výsledků
z důvodu značné aproximace, s níž algoritmus pracuje. O potenciálu zlepšení jsme
přesvědčeni, protože \citet{kmeans_layout} dosáhli podobným způsobem kvalitních
výsledků.

Dalším potenciálním rozšířením simulátoru je doimplementovat potřebné funkcionality
pro optimalizaci počtu nabíjecích slotů nabíjecích stanic a navržení vhodných
optimalizačních algoritmů řešící tuto problematiku.

Dále je v simulátoru částečně naimplementována funkcionalita pro načítaní a zápis
reprezentace nabíjecích stanic do souboru. Dokončení implementace by nám umožnilo
například optimalizovat již předpřipravené řešení a zkvalitnit tak optimalizaci,
neboť ve všech akuálně používaných optimalizační algoritmech inicializujeme 
optimalizaci na náhodně rozmístěných stanicích.

Také se nabízí potenciál v generování výjezdů a tras cest vozidel založených na 
reálných datech. Čímž bychom mohli zásadně zlepšit realističnost simulovaného
provozu. 