\chapter{Popis simulátoru}
\label{chap:popis_simulatoru}

V této kapitole definujeme řadu důležitých pojmů pro správný popis simulátoru
a dále popisujeme veškeré vlastnosti a funkcionality simulátoru.

Hlavní cílem návrhu simulátoru je uživatelská přívětivost v rámci analýzy
různých variant optimalizačních algoritmů. S ohledem na námi řešený problém je 
kladen důraz především na rozšířitelnost programu o nové optimalizační techniky
a snadnou nastavitelnost parametrů úzce souvisejících
s vlastnostmi nabíjecích stanic, mezi něž patří například počet nabíjecích stanic, 
rozmístění stanic, počet nabíjecích slotů na stanicích a podobně.

\section{Poznámka ke značení a definicím}
V následujícím textu definujeme řadu objektů simulace, jenž jsou často reprezentovány
uspořádanou množinou veličin. Často budeme potřebovat specifikovat jednotlivé 
složky těchto objektů. Pro zjednodušení notace, tak budeme zapisovat parametry
ve formátu \texttt{objekt.jméno\_parametru}. 

Tedy pokud například definujeme objekt $o$ jako uspořádanou množinu $(x, y, z)$.
Pak parametr $y$ objektu $o$ zapisujeme ve formátu $o.y$.

Dále je potřeba zmínit, že některé definice uvedené v následujícím textu, jsou
zjednodušeny a jejich zápis nemusí být matematicky naprosto přesný. 
Jsou například vynechány nějaké předpoklady z definic, které zřejmě vyplývají z 
textu. K tomuto kroku bylo přistoupeno pod snahou zpřehlednit jednotlivé definice.


\section{Předpoklady návrhu modelu}
\label{sec:predpoklady_navrhu}
S ohledem na výpočetní efektivitu simulátoru a uživatelskou přívětivost je potřeba
náležitě zjednodušit simulovaný model.

Dopravní síť reprezentujeme vhodným grafem rozšířeným o potřebné parametry. Simulátor je
navržen pro simulaci dopravních sítí pokrývající větší územní celky, které jsou 
rozlohou a hustotou sítě srovnatelné s dopravní sítí České republiky. Z tohoto
důvodu není zaručeno vhodné chování simulace na dopravních sítích s výrazně odlišnými
vlastnostmi.

Při volbě velikosti simulované oblasti jsme předpokládali, že většina majitelů
vozidel má možnost dobíjet své vozy v počáteční a cílové destinaci (např. mají
nabíjecí sloty v místě bydliště, či na pracovišti). Pro cesty na krátké vzdálenosti
tak budou schopni dobít své automobily na dostatečnou hladinu baterie a během cesty
nebude potřeba použít nabíjecí stanici. 

Na základě tohoto předpokladu je kladen důraz především na realistické simulování
dálkových tras vozidel. V zájmu výpočetní efektivity simulátoru a zkoumaného problému
opomíjíme většinu cest na krátkou vzdálenost a naopak nadhodnocujeme počet dálkových
tras vozidel. V simulátoru je nastavitelná střední hodnota vzdálenosti trasy, čímž
můžeme míru tohoto předpokladu náležitě regulovat.

Důsledkem zjednodušení je v simulaci zanedbána část vlastností reálné 
dopravní sítě. Například hustota provozu či skutečnost, že v reálném 
případě budou nabíjecí stanici využívat také řidiči jedoucí na krátké 
vzdálenosti tak nebude v simulaci dostatečně zahrnuta.

Protože je v reálném případě dopravní síť značně komplexní (především na 
úrovni silnic nižší třídy), je potřeba síť řádně zjednodušit. Tato problematika
je podrobněji popsána v kapitole o přípravě mapy viz. \cref{chap:priprava_mapy}.


\section{Dopravní síť}
\label{sec:dopravni_sit}

Řada myšlenek použitých pro definici modelu silniční sítě je inspirována 
návrhem modelu v práci \citet{kmeans_layout}.

Dopravní síť je v zájmu efektivity a jednoduchosti simulace reprezentována
vhodným grafem. K popisu grafu potřebujeme nejprve zadefinovat několik 
pomocných pojmů a objektů simulátoru.

Prvním pojmem je \emph{křižovatka}, jež zastupuje podmnožinu vrcholů v 
grafu silniční sítě, které reprezentují křižovatky v reálné silniční síti.

\begin{defn}[Křižovatka]\label{def:krizovatka}
    \emph{Křižovatka} je uspořádaná čtveřice $(n, c, m, d)$, kde:
    \begin{itemize}
        \item $n \in \mathbb{N}_0$ - identifikátor
        \item $c \in \mathbb{R}^2$ - zeměpisné souřadnice (viz. \cref{sect:zem_souradnice})
        \item $m \in \mathbb{N}_0$  - identifikátor nejbližšího města od křižovatky
        \item $d \in \mathbb{R}$ - vzdálenost vrcholu od středu města s 
        identifikátorem $m$
    \end{itemize}
\end{defn} 

Dále definujme \emph{typy silnice} používané simulátorem, s jejichž pomocí můžeme v
simulaci dopravy rozlišovat mezi různými kapacitami jednotlivých úseků dopravní sítě
(např. dálnice typicky efektivněji pokryjí hustý provoz než silnice 1. třídy).

\begin{defn}[Množina typů silnice]\label{def:mnoz_typu_silnice}
    \emph{Množina typů silnice} je množina $\{m, t, p, o\}$ všech typů 
    silnice používaných v simulaci dopravní sítě, kde:
    \begin{itemize}
        \item $m$ - dálnice
        \item $t$ - silnice pro motorová vozidla
        \item $p$ - silnice 1. třídy
        \item $o$ - ostatní typy silnic
    \end{itemize}
\end{defn}

\emph{Segmenty silnic} reprezentují vybrané body v silniční síti. A slouží jako
vrcholy grafu.

\begin{defn}[Segment silnice]\label{def:segment_silnice}
    \emph{Segment silnice} je uspořádaná dvojice $(n, b)$, kde:
    \begin{itemize}
        \item $n \in \mathbb{N}_0$ - identifikátor segmentu (určuje přesnou pozici)
        \item $b \in [0, 1]$ - průměrná hladina baterie vozidel v místě segmentu
    \end{itemize}
\end{defn}

\emph{Silnice} reprezentuje část silniční sítě mezi 2 křižovatkami, jenž neprochází
žádnou křižovatkou. Tento objekt shlukuje všechny segmenty silnice mezi dvojicemi
křižovatek.

\begin{defn}[Silnice]\label{def:silnice}
    Nechť $K$ značí množinu všech křižovatek v silniční síti a
    $T$ značí množinu typů silnic. Zároveň předpokládejme, že $k_1 < k_2$,  
    
    Pak \emph{silnice nad množinou křižovatek $K$} je uspořádaná
    pětice $(k_1, k_2, d, t, S)$, kde:
    \begin{itemize}
        \item $k_1, k_2 \in K$ - křižovatky na koncích silnice 
        \item $d \in \mathbb{R}^+$ - délka segmentu silnice
        \item $t \in T$ - typ
        \item $S$ - množina segmentů silnice, pro kterou je splněna podmínka: 
        $$(\forall{s, r} \in \mathbb{N}; r < s)(s \in S \implies r \in S)$$
    \end{itemize}
\end{defn}

Nyní již můžeme zadefinovat \emph{graf dopravní sítě}, s nímž pracujeme v naší simulaci
dopravy.

\begin{defn}[Graf dopravní sítě]\label{def:graf_site}
    Mějme množinu křižovatek $K$, množinu silnic $R$ nad množinou 
    křižovatek $K$ a $d \in \mathbb{R}$, pro které jsou splněny podmínky:
    \begin{enumerate}
        \item $$(\forall r_1, r_2 \in R; r_1 \neq r_2)\left(|r_1.S \cap r_2.S| \leq 1\right)$$
        \item $$(\forall r_1, r_2 \in R)\left(\{r_1.k_1, r_1.k_2\} 
            \cap \{r_2.k_1, r_2.k_2\} = \emptyset 
            \implies r_1.S \cap r_2.S = \emptyset \right)$$
        \item $$(\forall r \in R)(r.d = d)$$
    \end{enumerate}
    
    Definujme množinu segmentů všech silnic z $R$:
        $$V = \bigcup_{r \in R} r.S$$
    Následně definujme množinu sousedních dvojic segmentů silnice $r \in R$ jako:
        $$E_r = \{(s_1, s_2) \in (r.S)^2; |s_1.n - s_2.n| = 1\}$$
    A množinu sousedních dvojic segmentů všech silnic z $R$ jako:
        $$E = \bigcup_{r \in R} E_r$$

    Pak \emph{graf dopravní sítě} $G=(V, E)$ je uniformně ohodnocený souvislý 
    graf s vrcholy $V$ a hranami $E$ délky $d$.
\end{defn}

\section{Objekty dopravní sítě}

\subsection{Segmenty}
Segmenty silnice z \cref{def:segment_silnice} jsou nejmenší stavební jednotkou
celého simulátoru. Reprezentují vrcholy grafu silniční sítě a slouží k 
diskretizaci reálné silniční sítě, která je s jejich pomocí rozdělena 
na uniformně dlouhé úseky. Uchovává informaci o průměrné hladině baterie
v jednotlivých usecích dopravní sítě, která je využívána v optimalizaci.


\subsection{Křižovatky}
Křižovatky z \cref{def:krizovatka} zastupují podmnožinu vrcholů grafů
sítě, stupně vyššího než 2. Společně s městy se jedná o jediné objekty 
simulátoru, jenž si uchovávají zeměpisnou polohu v reálné dopravní síti.

Za pomocí těchto objektů jsou v simulátoru významným způsobem optimalizovány
algoritmy pro grafové hledání optimální cesty. Jsou používány k redukci vrcholů
a hran prohledávaného grafu, k heuristickému prohledávání stavového prostoru
při hledání nejkratších cest, či vhodných nabíjecích stanic. Jsou také hojně
používány v optimalizačních algoritmech pro rozmístění nabíjecích stanic.


\subsection{Silnice}

Silnice z \cref{def:silnice} sdružují podmnožinu hran grafu ležící mezi
2 křižovatkami a neprocházející jinou křižovatkou. Jsou potřebné pro 
zadefinování hran grafu simulátoru. Uchovávají informaci
o typu silnice, jsou používány pro hledání nejkratší cesty v grafu a jsou 
používány k simulování hustoty provozu v různých částech dopravní sítě.

V simulátoru předpokládáme, že délky všech silnic dopravní sítě jsou 
dělitelné délkou segmentu. To však v reálném případě typicky nenastane. V 
popisu reálné silniční sítě pracujeme pouze s přibližnou délkou jednotlivých
silnic. Tato skutečnost však výrazně neovlivňuje námi požadované vlastnosti
simulace.

\subsubsection{Hustota provozu}
Silnice je používána pro simulování hustoty provozu v jednotlivých úsecích dopravní
sítě. V simulátoru je zaimplementována jednoduchá logika pracující s hustotou
provozu v závislosti na typu (viz. \cref{def:mnoz_typu_silnice}) a délce 
silnice.

Princip simulování hustoty provozu je založen myšlence \emph{kapacity silnice}.

\begin{defn}[Kapacita silnice]\label{def:kapacita_silnice}
    Mějme $d$ délku hran grafu dopravní sítě $G$ z \cref{def:graf_site}.
    Nechť $R$ je množina silnic $G$. Dále $f:T \to \mathbb{R}$, kde $T$ je 
    množina všech typů silnic z \cref{def:silnice} a $\alpha, \beta \in \mathbb{R}$.

    \emph{Kapacita silnice} je funkce $c:R \to \mathbb{R}^+$, definována jako:
        $$c(r) := \alpha \cdot f(r.t) + \beta \cdot d \cdot |r.S|$$
\end{defn}

V simulátoru byly hodnoty $\alpha, \beta$ zvoleny na základě empirické volby. 
Simulování hustoty provozu je pak odvozeno od aktuálního počtu vozidel na 
silnici. Pomocí jednoduché funkce \emph{relativní délky silnice}.
 
\begin{defn}[Relativní délka silnice]\label{def:relativni_delka_silnice}
    Mějme $d$ délku hran grafu dopravní sítě $G$. Silnici $r$ grafu $G$,
    jejíž kapacita je $c_r \in \mathbb{R}^+$ a 
    nechť $n_r \in \mathbb{N}_0$ je aktuální počet vozidel na 
    silnici $r$. Dále funkce zpoždění vozidla $f:\mathbb{R} \to \mathbb{R}^+$.

    \emph{Relativní délka silnice} $r$ je definována, jako 
    $t_r:\mathbb{R}^+ \times \mathbb{N}_0 \to \mathbb{R}^+$, kde:
    $$
        t_{r}(n_r) =
          \begin{cases}
            |r.S| & pro\ n_r \leq c_r \\
            |r.S| \cdot f(n_r - c_r) & pro\ n_r > c_r
          \end{cases}       
    $$
\end{defn}

Doba průjezdu vozidla po silnici se pak vypočítá jako doba potřebná pro ujetí
vzdálenosti $d$, která je rovna relativní délce silnice, vozidlem s pevně
zvolenou průměrnou rychlostí $v$ (danou parametry simulátoru). Simulátor 
vždy aktualizuje hodnotu relativní délky silnice po změně počtu vozidel 
na silnici.


\subsection{Města}
Důležitým prvkem simulátoru je zakomponování vlivu zalidnění simulovaných oblastí.
K tomuto účelu nám slouží objekt \emph{město}.

\begin{defn}[Město]\label{def:mesto}
    \emph{Město} je uspořádaná trojice $(n, c, p)$, kde:
    \begin{itemize}
        \item $n \in \mathbb{N}_0$ - identifikátor
        \item $c \in \mathbb{R}^2$ - zeměpisné souřadnice centra města 
        (viz. \cref{sect:zem_souradnice})
        \item $p \in \mathbb{N}_0$ - počet obyvatel města
    \end{itemize}
\end{defn}

Každému městu náleží příslušná množina křižovatek definována jako:

\begin{defn}[Množina křižovatek města]\label{def:krizovatky_mesta}
    Mějme město $m$ v simulátoru a množinu všech křižovatek $K$ grafu $G$,
    pak \emph{množina křižovatek města} je:
    $$K_m = \{k \in K| k.m = m.n\}$$ 
\end{defn}

Koncept měst je důležitý pro realističtější generování počátečních
a koncových pozic vozidel v simulaci. Slouží také pro heuristický
výběr vhodné nabíjecí stanice pro nabití vozu.


\subsection{Nabíjecí stanice}
\emph{Nabíjecí stanice} jsou vzhledem ke zkoumanému problému důležitým 
prvkem simulátoru.

\begin{defn}[Nabíjecí stanice]\label{def:nabijeci_stanice}
    \emph{Nabíjecí stanice} v grafu dopravní sítě $G=(V, E)$ je uspořádaná 
    čtveřice $(n, m, v, c)$, kde:
    \begin{itemize}
        \item $n \in \mathbb{N}_0$ - identifikátor
        \item $m \in \mathbb{N}_0$ - identifikátor nejbližšího města od stanice
        \item $v \in V$ - pozice stanice (vrchol v grafu dopravní sítě)
        \item $c \in \mathbb{N}$ - kapacita stanice (počet zákazníků, 
        jež stanice pokryje najednou)
    \end{itemize}
\end{defn}

Informace o nejbližším městě je důležitá především pro heuristickou volbu 
nabíjecí stanice vozidlem v simulaci. V simulátoru je uchovávána informace
o aktuálním vytížení stanice a o vozidlech čekajících na volný nabíjecí slot pro 
simulování čekání automobilů ve frontě. Simulátor si pro každou nabíjecí 
stanici pamatuje průměrnou hladinu baterie vozidel při příjezdu do stanice,
pro použití v optimalizačních algoritmech. 
V návrhu simulátoru předpokládáme, že zákazníci stanice dobíjejí své vozidlo 
na plnou kapacitu baterie. Návrh simulátoru však umožňuje rozšíření o 
nabíjení pouze na zvolenou hladinu baterie bez významnějších zásahů do původní
implementace.


\section{Vozidla}
\label{sec:vozidla}

\emph{Vozidla} jsou nejdůležitějším prvkem simulace. Jednotlivá vozidla jsou 
simulátorem spravována samostatně. Každému vozidlu je při generování přiřazena
počáteční a cílová pozice a počáteční hladina baterie.


\subsection{Generování vozidel}
\label{subsec:generovani_vozidel}

Vozidla jsou v simulaci generována, aby co nejlépe aproximovala reálný provoz
vozidel. Doba čekání mezi výjezdy vozidel je náhodně generována z exponenciální
distribuce s nastavitelnými parametry. Počáteční hladina baterie vozidla je 
vygenerována z normálního rozdělení s pevně zvolenými parametry. Definujeme
také horní a dolní hranici hladiny počáteční baterie. Pokud vygenerovaná
hladina překračuje jednu z těchto hranic, pak je nahrazena hodnotou
překročené hranice.
Tento krok slouží jako ochrana pro vygenerování hladiny baterie pouze
v rozmezí $[0, 1]$.


Počáteční i cílová pozice vozidla je generována několikafázově z kombinace 
více náhodných distribucí. 

Při generování pozice výjezdu je nejprve generováno město $m_1$ výjezdu s 
pravděpodobností úměrnou jeho počtu obyvatel (čím více obyvatel, tím je 
výjezd vozidla pravděpodobnější). Tedy:

\begin{defn}[Pravděpodobnost výjezdu z města]\label{def:pravdepodobnost_mesta}
    Zvolme množinu $M$ všech měst v simulátoru. Pak 
    \emph{pravděpodobnost výjezdu} vozidla z města $m \in M$, je rovna:
        $$p_m = \frac{m.p}{\sum_{m_i \in M} m_i.p}$$
\end{defn}

Po vygenerování počátečního města $m_1$ je uniformně náhodně zvolena 
křižovatka $k_1$ z množiny křižovatek města $m_1$ 
(viz. \cref{def:krizovatky_mesta}). Následně je uniformně náhodně vybrána
silnice $r_1$ napojená na křižovatku $k_1$. Neboli silnice $r_1$, pro kterou
platí $ k_1 \in \{r_1.k_1, r_1.k_2\}$. Nakonec je uniformně náhodně zvolen
segment $v_1$ (vrchol grafu silniční sítě) silnice $r_1$, který již jednoznačně
popisuje pozici výjezdu vozidla. 

Generování cílové pozice vozidel je realizováno podobně s rozdílem generování 
koncového města. Při náhodném výběru koncového města je pravděpodobnost
výběru města ovlivněna počtem obyvatel města a vzdáleností od počátečního města.

Proces generování cílového města probíhá následovně. Nejprve je spočítána 
pravděpodobnost výjezdu vozidla pro všechna města v simulátoru 
(viz. \cref{def:pravdepodobnost_mesta}). Dále je pro počáteční město $m_1$
a každé město $m_i \in M$ simulátoru vypočítána jejich geografická vzdálenost
(viz. \cref{sect:zem_souradnice}), kterou označíme jako $d_{1,i}$. 
Následně pro všechna města $m_i \in M$ vypočítáme pravděpodobnost vzdálenosti $d_{1,i}$,
kterou si označíme $p_{1,i}$. Pravděpodobnost vzdálenosti $p_{1,i}$ vypočítáme
s pomocí nastavitelné exponenciální distribuce popisující vzdálenost cílové
pozice od počáteční pozice. Pravděpodobnost cílového města $m_2 \in M$
vůči počátečnímu městu $m_1 \in M$ spočítáme následovně:

\begin{defn}[Pravděpodobnost cílového města]\label{def:pravdepodobnost_ciloveho_mesta}
    Mějme $\alpha \in \mathbb{R}^+$ a počáteční město $m_1 \in M$. Označme vzdálenost 
    města $m_1$ od města $m_2 \in M$ jako $d_{m_1,m_2} \in \mathbb{R}$. 
    Označme pravděpodobnost
    výjezdu z města $m_2$ jako $p_{m_2}$ a pravděpodobnost vzdálenosti $d_{m_1,m_2}$ jako
    $p_{m_1,m_2}$, kterou dostaneme z námi zvolené exponenciální distribuce 
    (stejná notace platí pro libovolné město $m_i \in M$).
    Pak pravděpodobnost $p_{m_1,m_{2_{final}}}$, že město $m_2$ je cílovým městem
    vozidla s počátkem ve městě $m_1$ je rovna:
        $$p_{m_1,m_{2_{final}}} = 
            \frac{p_{m_2} + \alpha \cdot p_{m_1,m_2}}
            {\sum_{m_i \in M} p_{m_i} + \alpha \cdot p_{m_1,m_i}}$$
\end{defn}

Cílové město $m_2 \in M$ vygenerujeme z distribuce měst dané pravděpodobnostmi
cílového města vůči počátečními městu $m_1$ (viz. \cref{def:pravdepodobnost_ciloveho_mesta}).
Proces generování přesné cílové pozice ve městě $m_2$ je totožný s procesem 
generování přesné počáteční pozice ve zvoleném městě $m \in M$.

\begin{algorithm}
\begin{algorithmic}
\Function{Generování náhodné pozice}{$m$ - město}
    \State Uniformně náhodně vygeneruj křižovatku $k$ ve městě $m$.
    \State Uniformně náhodně vygeneruj silnici $r$ napojenou na křižovatku $k$.
    \State Uniformně náhodně vygeneruj segment $v$ na silnici $r$.
    \State \Return $v$
\EndFunction
\end{algorithmic}
\caption{Postup při náhodném generování pozice ve městě (vrcholu v grafu).}
\label{alg:generovani_pozice_ve_meste}
\end{algorithm}


\begin{algorithm}
\begin{algorithmic}
\Function{Generování vozidla}{}
    \State Z normálního rozdělení vygeneruj počáteční hladinu baterie $b$.
    \State V závislosti na počtu obyvatel náhodně vygeneruj počáteční město $m_1$.
    \State $v_1 \gets$ \Call{Generování náhodné pozice}{$m_1$}
    \State V závislosti na počtu obyvatel a vzdálenosti of města $m_1$ náhodně vygeneruj
    koncové město $m_2$
    \State $v_2 \gets$ \Call{Generování náhodné pozice}{$m_2$} 
    \State Z exponenciální distribuce vygeneruj čas $t$ čekání na výjezd vozidla.
    \State $vehicle \gets$ vozidlo s počáteční hladinou baterie $b$, 
    počátečním vrcholem $v_1$ a cílovým vrcholem $v_2$.
    \State \Return $vehicle$, $t$ \Comment vygenerovené vozidlo a čas čekání na jeho výjezd.

\EndFunction
\end{algorithmic}
\caption{Pseudokód procesu generování nového vozidla v simulaci.}
\label{alg:generovani_vozidla}
\end{algorithm}


\subsection{Hledání trasy vozidla}
\label{subsec:hledani_trasy}

Po vygenerování vozidla musíme v simulátoru vyřešit také otázku naplánování
trasy vozidla, které je realizováno algoritmem A-star (viz. \cref{sec:a_star}),
který jako heuristickou funkci volí zeměpisnou vzdálenost 
(viz. \cref{sect:zem_souradnice}) mezi křižovatkami.

S ohledem na efektivitu výpočtu pracujeme v simulaci při hledání cesty
se \emph{zjednodušeným grafem dopravní sítě} vůči počátečnímu vrcholu 
$v_1$ a cílovému vrcholu $v_2$.

\begin{defn}[Zjednodušený graf dopravní sítě]\label{def:zjednodusena_dopravni_site}
    Mějme graf dopravní sítě $G=(V, E)$ (viz. \cref{def:graf_site}) s množinou
    křižovatek $K$ a množinou silnic v síti $R$. Zvolme počáteční $v_1 \in V$ 
    a koncový vrchol $v_2 \in V$ a silnice $r_1 \in R$, resp. $r_2 \in R$, 
    t.ž. $v_1 \in r_1.S$, resp. $v_2 \in r_2.S$. 

    \emph{Zjednodušený graf dopravní sítě} vůči počátečnímu vrcholu $v_1$ a koncovému
    vrcholu $v_2$ je graf $H=(V', E')$, kde:
    \begin{itemize}
        \item $V' = K \cup r_1.S \cup r_2.S$
        \item $E' = (R \setminus \{r_1, r_2\}) \cup \{$hrany na silnicích $r_1$ a $r_2\}$
    \end{itemize}
\end{defn}

Zjednodušený graf dopravní sítě vůči počátečnímu vrcholu $v_1$ a cílovému vrcholu $v_2$
(viz. \cref{def:zjednodusena_dopravni_site}), je tedy graf, jehož vrcholy jsou křižovatky a
hrany jsou silnice, s vyjímkou počáteční a koncové silnice, jejichž reprezentace je
ponechána z původní \cref{def:graf_site} grafu dopravní sítě.

Postup plánování trasy vozidla je pak následující. Zvolíme vhodný zjednodušený graf dopravní
sítě a hrany grafu ohodnotíme relativní délkou silnice 
(viz. \cref{def:relativni_delka_silnice}), v případě počáteční a koncové silnice tuto
hodnotu náležitě rozdělíme mezi jednotlivé hrany na silnici. Následně na tomto grafu 
spustíme algoritmus A-star pro nalezení nejkratší cesty mezi vrcholy $v_1$ a $v_2$ s
ohledem na aktuální provoz v dopravní síti.

Algoritmus A-star používá jako heuristickou funkci zeměpisnou vzdálenost křižovatek. 
Neumíme však efektivně vypočítat tuto vzdálenost pro segmenty 
(viz. \cref{def:segment_silnice}). Při hledání nejkratší cesty se tedy omezujeme
pouze na hledání cest mezi nejbližšími křižovatkami počátečního a koncového vrcholu.
Výslednou cestu následně dostaneme řádným rozšířením cesty z počátečního a koncového
vrcholu do jejich nejbližších křižovatek. Vyjímkou je cesta uvnitř jedné silnice, kterou
nalezneme triviálně.

Vzhledem k výpočetní náročnosti hledání optimální cesty vozidla, hledáme tuto cestu
pouze při vygenerování vozidla, či ve speciálních případech, jako je například
plánování cesty na nabíjecí stanici, nebo cesty z nabíjecí stanice.


\subsection{Zajíždění na nabíjecí stanici}
\label{subsec:zajizdeni_na_nabijecku}

Speciální situace nastává v moment, kdy vozidlo potřebuje navštívit nabíjecí stanici.
V tomto případě je naplánována cesta z počáteční pozice do heuristicky zvolené
nabíjecí stanice postupem popsaným v části \cref{subsec:hledani_trasy}.
Po řádném nabití vozidla je pak stejným způsobem nalezena cesta ze stanice do
cílové pozice.

\subsubsection{Rozhodnutí o zajetí na nabíjecí stanici}

Každé vozidlo má možnost na základě aktuální hladiny baterie aktualizovat 
svou trasu tak, aby směřovala k nejbližší nabíjecí stanici. 
Z důvodu výpočetní náročnosti přepočítání trasy, může ale každé vozidlo
využít této možnosti pouze jednou. Po zajetí vozidla na nabíjecí stanici je mu
však tato možnost znovu přidělena.

Jelikož každé vozidlo může pozměnit svou trasu pouze jednou, je důležité zvolit
vhodný okamžik přeplánování trasy. Tento okamžik naplánujeme pomocí vhodně 
nastavitelného normálního rozdělení, ze kterého vygenerujeme hladinu baterie,
při které se v budoucnu rozhodneme zda zamíříme na nabíjecí stanici, či ne.

Volby okamžiku rozhodování s pomocí normálního rozdělení je motivován následujícími
myšlenkami. Nechceme vyrazit ani plánovat cestu na nabíjecí stanici moc brzy, neboť se 
může v průběhu cesty podstatně změnit hustota provozu a naplánovaná cesta
tedy nebude optimální. Zároveň je cesta na nabíjecí stanici při dostatečně vysoké 
hladině baterie zbytečná z důvodu malé kapacity baterie, kterou je možno doplnit.
Naopak v případě velmi nízké hladiny baterie, při níž aktualizujeme cestu, můžeme
narazit na problém, že vozidlo nedojede na nabíjecí stanici před úplným vybitím baterie.

Zda vozidlo v moment aktualizace cesty vyrazí na nabíjecí stanici, rozhodneme
na základě očekávané hladiny baterie vozidla v cílové destinaci, kterou vypočítáme
z aktuální hustoty provozu a spotřeby vozidla. V simulátoru je nastavitelná 
nejnižší očekávaná cílová hladina baterie, pro kterou vozidlo ještě nemění svou trasu,
tak aby směřovalo na nabíjecí stanici.


\subsubsection{Výběr nabíjecí stanice}

Vozidlo se musí v okamžiku rozhodnutí vyrazit na nabíjecí stanici rozhodnout,
kterou stanici zvolit. Toto rozhodnutí je realizováno heuristicky. 

Postup výběru stanice probíhá následovně. Pro každé město (viz. \cref{def:mesto})
je na začátku simulace nalezen námi zvolený počet očekávaných nejbližších 
nabíjecích stanic. Množinu těchto stanic pro libovolné město $m$ označíme 
jako $C_m$. 

Postup heuristického výběru nejbližších stanic města je následující. Pro každou stanici
$c$ zvolíme její nejbližší křižovatku $k$ (viz. \cref{def:krizovatka}) a město $m_k$,
do nějž křižovatka $k$ náleží (viz. \cref{def:krizovatky_mesta}). 
Vzdálenost stanice $c$ od libovolného města $m$ pak spočítáme jako součet $k.d$ a
zeměpisné vzdálenosti (viz. \cref{sect:zem_souradnice}) měst $m$ a $m_k$.

Následně pak v závislosti na městu $m$, kterému náleží křižovatka $k$,
která je nejblíže od aktuální pozice vozidla, zvolíme kandidáty na vhodné 
nabíjecí stanice pro zvolené vozidlo. Tito kandidáti jsou právě stanice 
$C_m$. Pro každého kandidáta $c \in C_m$ následně vypočítáme s pomocí relativních délek 
silnic (viz. \cref{def:relativni_delka_silnice}) očekávanou dobu cesty 
(délka cesty v grafu relativních délek) $d_c$ vedoucí z aktuální pozice přes
nabíjecí stanici $c$ do cílové pozice.
Zvolená nabíjecí stanice $c_{min} \in C_m$, je taková, 
že $d_{c_{min}}$ je nejmenší ze všech stanic z $C_m$.

\begin{algorithm}
\begin{algorithmic}
\Function{Výběr nabíjecí stanice}{\newline$v_{start}$ - aktuální pozice vozidla,\newline
    $v_{end}$ - cílová pozice vozidla,\newline
    $C_{closest}$ - množina množin nejbližších stanic každého města}
    \State $k \gets$ Nejbližší křižovatka od $v_{start}$.
    \State $C_{closest_m} \gets$ Množina z $C_{closest}$ náležící městu s ID $k.m$
    \State $d_{min} \gets 0$ \Comment Nejkratší očekávaná doba cesty.
    \State $c_{min}$ \Comment Aktuálně nejlepší stanice pro výběr. 
    \ForAll{$c \in C_{closest_m}$}
        \State $d_{in} \gets$ \Call{A-star}{$v_{start}$, $c.v$} 
            \Comment Očekávaná doba cesty do stanice.
        \State $d_{out} \gets$ \Call{A-star}{$c.v$, $v_{end}$}
            \Comment Očekávaná doba cesty ze stanice do cíle.
        \If{$d_{in} + d_{out} < d_{min}$ nebo $c_{min}$ není definováno}
            \State $c_{min} \gets c$ \Comment Aktualizace nejlepší stanice.
            \State $d_{min} \gets d_{in} + d_{out}$
        \EndIf
    \EndFor
    \State \Return c \Comment Předej vybranou nabíjecí stanici.
\EndFunction
\end{algorithmic}
\caption{Proces výběru nabíjecí stanice vozidlem.}
\label{alg:vyber_stanice}
\end{algorithm}

Tento zjednodušený přístup hledání vhodné nabíjecí stanice podstatně zefektivňuje
simulaci. Neboť výpočetně nejnáročnější operací celé simulace je hledání nejkratších
cest v grafu. Omezením výběru stanic rychlost simulace výrazně vzroste. 


\subsubsection{Nakládání s vybitými vozidly}

Simulátor musí řešit také případy, kdy se vozidlům vybije baterie a nemůžou 
dojet do cíle. V takovém případě je vozidlo vyřazeno ze simulace a dále již nijak
neovlivňuje simulaci. Veškeré informace o vozidlech s vybitými bateriemi jsou
však zaznamenávány a uchovávány pro algoritmy optimalizující pozice 
nabíjecích stanic.

\subsection{Druhy vozidel}
\label{subsec:druhy_vozidel}

Rozhraní simulátoru je přizpůsobeno možnosti rozlišovat mezi různými druhy 
vozidel, jejichž vlastnosti a chování na silnici se může lišit. V aktuální 
implementaci je použiván jediný typ vozidla pojmenovaný \emph{auto}.
Veškeré vlastnosti vozidel ze \cref{sec:vozidla}, platí pro tento typ
vozidla. Tento typ vozidla je specifický svým chováním po dosažení cílové 
destinace. Poté co auto dorazí do cíle čeká náhodně dlouhou dobu, jež je dána 
exponenciální distribucí a následně vyrazí zpět do počáteční pozice.

Očekáváme, že tento typ vozidla vhodně popisuje reprezentativní vzorek
vozidel, protože předpokládáme, že většina řidičů se typicky po čase vrací do místa
výjezdu. 


\section{Specifika reálné dopravní sítě}

V této sekci poukážeme na specifické případy při simulování reálné dopravní sítě,
které se odlišují od návrhu simulátoru, a odlišnosti v reálné implementaci simulátoru.

\subsection{Poznámky k definici modelu dopravní sítě}
\label{subsec:problemy_modelu}

V obsahu \cref{sec:dopravni_sit} popisujeme zjednodušený model dopravní sítě. Část
předpokladů, či vlastností zde popisovaných však plně neodpovídá realitě. Tyto
nesrovnalosti popisujeme v následujícím textu. 

Při popisu křižovatek tvrdíme, že se jedná o vrcholy stupně více než 2. 
Tento předpoklad je běžně
splněn i v reálné silniční síti. V implementaci simulátoru jsou reprezentovány reálnými 
křižovatkami. Ze sítě simulátoru jsou ale typicky odstraněny silnice nižších tříd, čímž
se také v grafu zmenšují stupně vrcholů a může docházet k degeneraci křižovatek
na vrcholy stupně 2 (viz. \cref{sec:extrakce_site}). Během přípravy mapy se 
snažíme tyto křižovatky z implementace eliminovat. Nezaručujeme však, že se v síti 
simulátoru nevyskytují žádné degenerované křižovatky stupně 2. 

Tato skutečnost ale zásadně neovlivňuje vlastnosti simulátoru. Hlavní motivací pro
zadefinování objektů křižovatek je zefektivnění hledání optimální trasy vozidel
(viz. \cref{subsec:hledani_trasy}), jež je s přehledem výpočetně nejnáročnější 
operací v simulátoru. Na reálné dopravní síti je redukce vrcholů stupně 2 
nepostradatelná s ohledem na praktickou použitelnost simulátoru na sítích větších 
územních celků.

\subsubsection{Problematika souvislosti grafu}
\label{subsubsec:souvislost_grafu}

Z \cref{def:graf_site} vyplývá, že graf dopravní sítě musí být souvislý graf. Vlivem
výběru pouze části dopravní sítě (více viz. \cref{sec:extrakce_site}) však typicky
dochází k rozpadu grafu sítě na více komponent souvislosti. Ty musíme před začátkem 
simulace dopravy sloučit s co nejmenším zásahem do reprezentace dopravní sítě.

Komponenty souvislosti slučujeme jednoduchou heuristickou metodou, založenou na 
hledání křižovatek nejbližších měst (viz. \cref{def:krizovatky_mesta}) 
z dvojice různých komponent. Tyto křižovatky následně propojíme silnicí typu
s nejnižší kapacitou (viz. \cref{def:mnoz_typu_silnice} a \cref{def:kapacita_silnice}) a
délkou danou zeměpisnou vzdáleností (viz. \cref{sect:zem_souradnice}).

Předpokládáme, že většina komponent souvislosti vznikla vlivem výřezu nebo úpravy mapy 
v hraničních či v málo osídlených oblastech. Proto předpokládáme, že přidání nové uměle 
vytvořené hrany zásadně neovlivňuje realističnost simulace reálné dopravní sítě.


\subsubsection{Poznámka ke generování vozidel}

Při generování vozidel (viz. \cref{subsec:generovani_vozidel}) pracujeme pouze
s hustotou zalidnění oblasti, což ale nutně nemusí realisticky popisovat reálný provoz.
V simulaci například nebereme v úvahu, že některé málo zalidněné oblasti mohou
být frekventovanější, než ty s vyšší hustotou zalidnění. Může se například jednat o
pracoviště mnoha zaměstnanců mimo obytnou oblast.

Při generování cílových měst zároveň předpokládáme, že přesun vozidel mezi velmi 
vzdálenými městy je málo pravděpodobný, ale zároveň se snažíme eliminovat cesty na
krátkou vzdálenost (typicky na území jednoho města) z důvodů zefektivnění simulace
a získání relevantních dat potřebných pro náš optimalizační problém.

V rámci realistické simulace je v oblasti generování vozidel vysoký potenciál 
pro budoucí zlepšení našeho simulátoru. Například můžeme použít metodu využitou 
v práci autorů \citet{kmeans_layout}, pracující s mírou zaměstnanosti v různých částech
simulované oblasti.

Simulátor je však navržen tak, aby bylo možno v budoucnu doimplementovat různé 
varianty generování vozidel s co nejmenším zásahem do již naimplementovaných 
variant. 


\subsubsection{Poznámka k hledání cest}

Popis hledání optimální cesty vozidla ze \cref{subsec:hledani_trasy} umožňuje
pouze jednu korekci během trasy. K tomuto kroku bylo přistoupeno vzhledem ke snaze
simulovat co nejhustší provoz. 

Je možné, že se hustota provozu v průběhu cesty vozidla výrazně změní a nalezená
trasa vozidla se bude výrazně lišit od reálně nejoptimálnější trasy. 
Vzhledem k malé hustotě simulovaného provozu, který zásadním způsobem omezujeme
pouze na dálkové trasy (viz. \cref{sec:predpoklady_navrhu}). Předpokládáme,
že tato situace nebude nastávat velmi často. 

Dalším odůvodněním přístupu plánování cesty je i skutečnost, že většina řidičů
plánuje své cesty před výjezdem a jen v omezené míře zásadně upravují svou 
trasu v průběhu jízdu, aby byla co nejoptimálnější. Jelikož většina řidičů není
schopna tento velice komplexní problém efektivně vyřešit v průběhu cesty. 
Jsme si vědomi možnosti používání široké škály navigačních programů, jež jsou 
schopny informovat řidiče o dopravní situaci a aktualizovat optimální cestu.
Předpokládáme však, že tyto zmíněné systémy nejsou neomylné a často není 
navržená cesta nejoptimálnější a náš zjednodušený přístup plánování cesty 
není výrazně horší ani rozdílný od reálného chování řidičů. 