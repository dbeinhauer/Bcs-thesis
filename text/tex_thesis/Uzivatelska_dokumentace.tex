\chapter{Uživatelská dokumentace}
\label{chap:uzivatelska_dokumentace}

Hlavním cílem programu je na zvolené reprezentaci mapy opakovaně simulovat 
dopravní síť a na základě této simulace optimalizovat rozmístění nabíjecích
stanic s pomocí různých optimalizačních algoritmů. Po ukončení optimalizace
vypíše zvolené relevantní informace o jednotlivých simulacích pro následnou 
analýzu optimalizačních metod. Program také nabízí nastavitelnost široké škály 
parametrů simulátoru dopravy i optimalizačních  metod pro detailnější analýzu.


\section{Příprava před spuštěním}
Rozlišujeme více variant spuštění programu podle OS, který používáme.

Pokud pracujeme v OS \textbf{Windows 10}, pak můžeme použít již předkompilovaný 
program, jenž je součástí přílohy. V tomto případě můžeme program rovnou spustit.
Předkompilovaný spustitelný se nachází v adresáři:

\begin{verbatim}
    Win_executable/
\end{verbatim}

V tomto adresáři se nachází spustitelný program, adresář s předpřipravenou
reprezentací mapy a další pomocné soubory pro správné fungování programu.
Název spustitelného programu je:

\begin{verbatim}
    Traffic_Simulator.exe 
\end{verbatim}

\subsection{Sestavení programu}
Pokud chceme program sestavit, pak je potřeba postupovat v následujících krocích.

\subsubsection{Sestavení pro OS Linux}

Pro sestavení programu v OS Linux potřebujeme
\texttt{CMake 3.11}\footnote{\url{https://cmake.org/}}, 
nebo vyšší (doporučujeme verzi \texttt{3.14+}), překladač podporující \texttt{C++11},
knihovnu \texttt{Boost}\footnote{\url{https://www.boost.org/}} a
Git\footnote{\url{https://git-scm.com/}}.

Nejprve se přesuneme do adresáře s požadovanými zdrojovými kódy.
\begin{verbatim}
    cd source_code/Traffic_Simulator_cmake
\end{verbatim}

Pro konfiguraci spustíme příkaz (přidáme ještě možnost \texttt{-GNinja}, 
pokud používáme Ninja\footnote{\url{https://ninja-build.org/}}).
\begin{verbatim}
    cmake -S . -B build
\end{verbatim}

Pro sestavení programu spustíme příkaz:
\begin{verbatim}
    cmake --build build
\end{verbatim}

Výsledný spustitelný program pak nalezneme:
\begin{verbatim}
    build/apps/Traffic_Simulator
\end{verbatim}

\subsubsection{Sestavení pro OS Windows 10}
Pro sestavení programu v OS Windows 10 potřebujeme překladač podporující 
\texttt{C++11}, Visual Studio 2019\footnote{https://visualstudio.microsoft.com/cs/},
Git a \texttt{vcpkg}\footnote{\url{https://github.com/microsoft/vcpkg}}.

Nejprve se přesuneme do adresáře, kde je uložen nástroj \texttt{vcpkg} a 
nainstalujeme knihovnu Boost.
\begin{verbatim}
    cd $(Absolute_Path_To_vcpkg_Directory)

    .\vcpkg\vcpkg install boost
    .\vcpkg\vcpkg integrate install
\end{verbatim}

Následně se přesuneme do adresáře naší práce a zde do adresáře:

\begin{verbatim}
    cd source_code\Traffic_Simulator_VS
\end{verbatim}
 
Zde nalezenem soubor:
\begin{verbatim}
    Traffic_Simulator.sln
\end{verbatim}

Ten otevřeme v programu Visual Studio 2019, ve kterém projekt sestavíme.

\section{Poznámka k příkladům}
V následujícím textu popisujeme příklady použití programu v OS Linux.
Pokud pracujeme v OS Windows, pak program spouštíme následujícím příkazem
(se zvolenými parametry (viz. \cref{subsec:parametry_programu})):

\begin{verbatim}
    .\Traffic_Simulator.exe
\end{verbatim}

Pokud pracujeme s předpřipravenou mapou z adresáře:
\begin{verbatim}
    preprocesed\_maps
\end{verbatim}

Pak musíme pro fungování příkazů v příkladech, zkopírovat tento adresář do 
adresáře se spustitelným souborem programu. Pokud pracujeme s již
předkompilovaným programem v OS Windows, pak tato operace není nutná.

\section{Formát souborů reprezentujících mapu}
\label{sec:soubor_mapy}

Pro správné fungování programu je potřeba při každém spuštění načíst reprezentaci
mapy (\cref{sec:dopravni_sit}).
Ta je rozdělena do 3 souborů a informace v nich zapsané si nesmějí 
navzájem odporovat. Tyto 3 soubory jsou načítany v pořadí: soubor s reprezentací
křižovatek, soubor s reprezentací silnic a soubor s reprezentací měst.
Vzhledem k obvykle velké objemnosti dat reprezentující dopravní síť je 
doporučeno při přípravě vstupních souborů postupovat podle
kroků, jež popisuje \cref{chap:priprava_mapy}, s pomocí přiložených programů
v adresáři \texttt{map\_preprocessing} práce (viz. \cref{chap:prilohy}).

Pokud chce uživatel pracovat pouze s námi připravenou mapou České republiky, 
jenž podrobněji popisuje \cref{chap:priprava_mapy}, je v adresáři
\texttt{preprocesed\_maps} dodána reprezentace již připravené mapy.
Struktura adresáře je:

\begin{verbatim}
./prepared_graph/
    cities.txt
    combined_nodes.txt
    edges.txt
    prepared_combined_nodes.txt
    prepared_edges.txt
\end{verbatim}

V souborech výše jsou zapsány po řadě, reprezentace měst, reprezentace křižovatek,
reprezentace hran a v souborech s předponou \texttt{prepared\_} je uložena 
předpřipravená reprezentace mapy (bez křižovatek stupně 2) se stejnou reprezentací,
jako v neupravené variantě (reprezentace měst se předpřípravou nemění).

Formát všech vstupních souborů je strukturován po řádcích. V každém souboru
je vždy ignorován první řádek, jehož účelem je popis formátu vstupních dat 
jednotlivých řádků pro každý vstupní soubor. S ohledem na 
přehlednost vstupních dat je doporučeno tento formát vždy dodržovat.
Ve zbytku souboru je postupně na každém řádku reprezentován specifický objekt
mapy, který je určen daným typem souboru. Jakékoli nedodržení formátu
vstupních souborů vede s vysokou pravděpodobností k selhání načítání dat a
není tak umožněna další práce s programem.


\subsection{Soubor reprezentující křižovatky}
\label{subsec:reprezentace_krizovatky}

Předpokládaná hlavička (1. řádek) souboru je:
\begin{Verbatim}
    node_id latitude longitude city_id distance_km 
\end{Verbatim}

Každý řádek, kromě prvního řádku, reprezentuje právě jednu kžižovatku (\cref{def:krizovatka}).
Požadujeme, aby na každém řádku bylo uloženo pět hodnot oddělených mezerou, které 
reprezentují po řadě: identifikační číslo (ozn. ID) křižovatky, zeměpisnou šířku,
v níž se křižovatka nachází, zeměpisnou délku, v níž se křižovatka nachází,
ID města, do něhož křižovatka náleží, a vzdálenost křižovatky od centra tohoto města.

Pro správnou funkci programu je potřeba, aby ID křižovatek byla celá nezáporná 
čísla a aby křižovatky byly v souboru seřazeny vzestupně bez vynechaných hodnot ID.
Tedy pokud např. v souboru existuje vrchol s ID 6, musí být předem definovány
vrcholy s ID v rozmezí od 1 do 5. Dále požadujeme, aby ID křižovatek byly unikátní.
Zeměpisná šířka, zeměpisná délka i vzdálenost města jsou reprezentovány 
desetinným číslem s desetinnou tečkou. ID města je reprezentováno nezáporným
celým číslem, jež musí být zadefinováno v souboru reprezentující města 
(\cref{subsec:reprezentace_mesta}).


Příklad formátu vstupního souboru, jenž reprezentuje křižovatky:
\begin{Verbatim}
    node_id latitude longitude city_id distance_km
    0 50.0354962 14.4080276 1 5.855519633308941 
    1 50.0352373 14.4080465 2 5.883717272182215 
    2 50.0350876 14.4080758 0 5.8998150943334124
    3 50.035876 14.4080352 0 5.8898140943334124
\end{Verbatim}


\subsection{Soubor reprezentující silnice}

Předpokládaná hlavička (1. řádek) souboru je:
\begin{Verbatim}
new_node_id_1 new_node_id_2 length road_type
\end{Verbatim}
Každý řádek, kromě prvního řádku, reprezentuje právě jednu silnici (\cref{def:silnice})
dopravní sítě. Požadujeme, aby na každém řádku byly uloženy čtyři hodnoty oddělené
mezerou reprezentující: ID první křižovatky definující silnici, ID druhé
křižovatky definující silnici, délka silnice a typ silnice.
Předpokládáme, že ID křižovatek jsou v odpovídajícím formátu, 
který je popsán v části pro reprezentaci křižovatek (\cref{subsec:reprezentace_krizovatky})
a zároveň, že křižovatka s ID ze vstupu byla definována a načtena ze souboru 
pro reprezentaci křižovatek.
Délka silnice je reprezentována desetinným číslem s desetinnou tečkou.
\Cref{tab:typy_silnice} vypisuje všechny typy silnice a jejich reprezentaci pro
vstupní soubor.

\begin{table}
\centering\footnotesize\sf
\begin{tabular}{ll}
\toprule
Reprezentace & Popis \\
\midrule
\texttt{m} & dálnice  \\
\texttt{t} & silnice pro motorová vozidla \\
\texttt{p} & silnice 1. třídy nebo  \\
\texttt{o} & jiný druh silnice \\
\bottomrule
\end{tabular}
\caption{Výpis všech variant druhů silnice.}
\label{tab:typy_silnice}
\end{table}
    

Příklad formátu vstupního souboru, jenž reprezentuje silnice:
\begin{Verbatim}
new_node_id_1 new_node_id_2 length road_type
0 1 28.82 t 
1 2 16.777 p
2 3 17.303 m
\end{Verbatim}


\subsection{Soubor reprezentující města}
\label{subsec:reprezentace_mesta}

Předpokládaná hlavička (1. řádek) souboru je:
\begin{Verbatim}
city_id lat lon population
\end{Verbatim}
Každý řádek, kromě 1. řádku, reprezentuje právě 1 město (\cref{def:mesto}).
Požadujeme, aby na každém řádku byly uloženy 4 hodnoty oddělené mezerou reprezentující:
ID města, zeměpisnou šířku, v níž se nachází střed města, zeměpisnou délku, v 
níž se nachází střed města, a počet obyvatel města.
Požadujeme, aby ID města bylo nezáporné celé číslo. Pro zajištění správného 
fungování programu je doporučeno reprezenovat města v souboru v rostoucím
pořadí podle ID města bez opakování a bez vynechaných ID. Tato podmínka 
však není na rozdíl od reprezentace křižovatek nutná.
Zeměpisná šířka a délka jsou reprezentovány desetinným číslem s desetinnou 
tečkou. Populace města je reprezentována nezáporným celým číslem.

Příklad formátu vstupního souboru, jenž reprezentuje  města:
\begin{Verbatim}
city_id lat lon population
0 49.39606475830078 15.590306282043457 51216 
1 49.70147705078125 17.07569122314453 7516 
2 51.02574920654297 15.061005592346191 4259 
\end{Verbatim}


\subsection{Výstupní soubory}

Program také umožňuje zápis aktuálně načtené mapy do souborů ve vstupním formátu
simulátoru. Tato funkcionalita je především užitečná, pokud je předpříprava
původní mapy výpočetně náročná (odstraňování vrcholů stupně 2 nebo 
sjednocování komponent souvislosti). V takovém případě můžeme předpřipravenou
mapu uložit do formátu určeného pro načítání mapy a při opakovaném výpočtu
přeskočit předpřípravu dat a rovnou spustit požadovaný výpočet.


\section{Ovládání programu}

Uživatel interaguje s programem pouze při jeho spuštění, kdy je potřeba
zvolit příslušnou optimalizační metodu, její parametry a parametry simulace.
Před spuštěním programu je potřeba připravit soubory s korektní 
reprezentací mapy (\cref{sec:soubor_mapy}) a specifikovat 
cestu k těmto souborům.


\subsection{Parametry programu}
\label{subsec:parametry_programu}
Při spuštění programu zadáváme do terminálu parametry, kterými specifikujeme
požadované chování simulátoru. Tyto parametry zapisujeme ve formě přepínačů za
příkaz spouštějící program. Tyto přepínače popisujeme v tabulkách rozdělených
podle jejich funkce.
\Cref{tab:prepinace_pomocne} popisuje pomocné přepínače programu, 
\cref{tab:prepinace_soubory} popisuje přepínače pro specifikování vstupních a 
výstupních souborů, \cref{tab:prepinace_simulace} popisuje přepínače pro volbu
parametrů simulace, \cref{tab:prepinace_optimalizace} popisuje přepínače pro 
volbu obecných vlastností typických pro všechny optimalizační metody,
\cref{tab:prepinace_greedy} popisuje přepínače pro volbu parametrů hladové optimalizace,
\cref{tab:prepinace_genetic}, popisuje přepínače pro volbu parametrů optimalizace genetickým
algoritmem, \cref{tab:prepinace_kmeans} popisuje přepínače pro volbu parametrů optimalizace
pomocí algoritmu k-means a \cref{tab:prepinace_volba_optim} popisuje přepínače pro volbu
optimalizační metody.

\begin{table}
\centering\footnotesize\sf
\begin{tabular}{p{0.15\linewidth} p{0.75\linewidth}}
\toprule
Příkaz & Popis funkce \\
\midrule
\texttt{help} & Vypíše všechny možnosti parametrů programu. \\
\texttt{logs} & Program bude vypisovat podrobné informace o průběhu simulace 
(pro ladění progamu). \\
\bottomrule
\end{tabular}
\caption{Popis pomocných přepínačů v programu.}
\label{tab:prepinace_pomocne}
\end{table}


\begin{table}
\centering\footnotesize\sf
\begin{tabular}{p{0.30\linewidth} p{0.6\linewidth}}
\toprule
Příkaz & Popis funkce \\
\midrule
\texttt{nodeCityFile arg} & Cesta k vstupnímu souboru s reprezentací křižovatek. 
V případě nespecifikované hodnoty je zvolena cesta:
\texttt{prepared\_graph/combined\_nodes.txt} \\
\texttt{edgesFile arg} & Cesta k vstupnímu souboru s reprezentací silnic.
V případě nespecifikované hodnoty je zvolena cesta:
\texttt{prepared\_graph/edges.txt} \\
\texttt{citiesFile arg} & Cesta k vstupnímu souboru s reprezentací měst. 
V případě nespecifikované hodnoty je zvolena cesta:
\texttt{prepared\_graph/cities.txt} \\
\texttt{addedEdgesFile arg} & Cesta k výstupnímu souboru pro uložení reprezentace 
přidaných hran pro vytvoření souvislého grafu. 
V případě nespecikované hodnoty je zvolena cesta:
\texttt{prepared\_graph/added\_edges.txt} \\
\texttt{preparedNodesFile arg} & Cesta k výstupnímu souboru pro uložení reprezentace
vrcholů upraveného grafu po eliminaci vrcholů stupně 2.
V případě nespecikované hodnoty je zvolena cesta:
\texttt{prepared\_graph/prepared\_combined\_nodes.txt} \\
\texttt{preparedEdgesFile arg} & Cesta k výstměupnímu souboru pro uložení reprezentace
hran upraveného grafu po eliminaci vrcholů stupně 2 
V případě nespecikované hodnoty je zvolena cesta:
\texttt{prepared\_graph/prepared\_edges.txt}\\
\texttt{savePrepared} & Parametr specifikující, zda uložit předpřipravenou 
reprezentaci mapy. \\
\bottomrule
\end{tabular}
\caption{Popis přepínačů pro specifikování vstupních a výstupních souborů.}
\label{tab:prepinace_soubory}
\end{table}


\begin{table}
\centering\footnotesize\sf
\begin{tabular}{p{0.43\linewidth} p{0.5\linewidth}}
\toprule
Příkaz & Popis funkce \\
\midrule
\texttt{simulationTime arg} & Doba simulace v minutách. \\
\texttt{segmentLength arg} & Délka hrany grafu v kilometrech. \\
\texttt{numClosestStations arg} & Počet nabíjecích stanic, které jsou zvažovány 
    při plánování cesty vozidla na nabíjecí stanici. \\
\texttt{carConsumption} & Spotřeba baterie vozidla v procentech kapacity za minutu. \\
\texttt{numStations arg} & Celkový počet nabíjecích stanic v simulaci. \\
\texttt{stationCapacity arg} & Počet nabíjecích slotů na každé stanici. \\
\texttt{exponentialLambdaCities arg} & Parametr pro generování cílového města vozidla.
    Hodnota je rovna $\frac{1}{d}$, kde $d$ je střední hodnota vzdálenosti 
    cílového města od startu v kilometrech. \\
\texttt{exponentialLambdaDepartures arg} & Parametr pro generování času výjezdu
    následujícího vozidla. Hodnota je rovna $\frac{1}{t}$, kde $t$ je střední 
    hodnota času následujícího výjezdu vozidla v minutách. \\
\texttt{endCityRatio arg} & Parametr specifikující důležitost vzdálenosti cílového
    města vůči počtu obyvatel při generování cílového města 
    (viz. \cref{def:pravdepodobnost_ciloveho_mesta}). Pokud $<1$, pak je
    počet obyvatel důležitější. Pokud $=1$, pak jsou oba parametry stejně důležité.
    Pokud $>1$, pak je důležitější vzdálenost města. \\
\texttt{batteryTresholdLambda arg} & Hodnota $\frac{1}{b}$, kde $b$ je střední 
    hodnota hladiny baterie specifikující, o kolik průměrně převyšuje hladina baterie v době
    rozhodnování nejnižší možnou hranici pro rozhodnutí, zda jet nabíjet, v moment rozhodnutí
    (viz. \cref{subsec:zajizdeni_na_nabijecku}). \\
\texttt{carBatteryMean arg} & Střední hodnota počáteční hladiny baterie. \\
\texttt{carBatteryDeviation arg} & Směrodatná odchylka počáteční hladiny baterie vozidla. \\
\texttt{carStartBatteryBottomLimit arg} & Minimální hladina baterie vozidla při výjezdu. \\
\texttt{chargingTreshold arg} & Nejnižší možná hladina baterie pro rozhodnutí,
    zda vyrazit na nabíjecí stanici (viz. \cref{subsec:zajizdeni_na_nabijecku}). \\
\texttt{notChargingTreshold arg} & Nejvyšší hladina pro rozhodnutí, zda vyrazit na 
    nabíjecí stanici (viz. \cref{subsec:zajizdeni_na_nabijecku}). \\
\texttt{batteryTolerance arg} & Nejnižší možná očekávaná hladina baterie v cíli, 
    kdy se při rozhodování, rozhodnout nezajet na nabíjecí stanici 
    (viz. \cref{subsec:zajizdeni_na_nabijecku}). \\
\texttt{carVelocity arg} & Průměrná rychlost všech vozidel v kilometrech za minutu. \\
\texttt{chargingWaitingTime arg} & Doba čekání na úplné nabití baterie v minutách.  \\
\texttt{meanChargingLevel arg} & Očekávaná střední hodnota hladiny baterie, jež
    je nabíjena na nabíjecích stanicích (pro odhad doby čekání nabíjení ve frontě).  \\
\bottomrule
\end{tabular}
\caption{Popis přepínačů pro specifikování parametrů simulátoru.}
\label{tab:prepinace_simulace}
\end{table}


\begin{table}
\centering\footnotesize\sf
\begin{tabular}{p{0.42\linewidth} p{0.51\linewidth}}
\toprule
Příkaz & Popis funkce \\
\midrule
\texttt{lossDifferenceTreshold arg} & Minimální rozdíl ztrátové funkce pro 
pokračování v optimalizaci (pro vybrané optimalizační metody). \\
\texttt{stationNumberParameter arg} & Parametr počtu stanic při výpočtu ztrátové 
funkce (viz. \cref{sec:loss}).\\
\texttt{runDownParameter arg} & Parametr počtu vozidel, kterým se vybila baterie,
pro výpočet ztrátové funkce (viz. \cref{sec:loss}).\\
\texttt{durationParameter arg} & Parametr průměrné doby cestování v minutách pro
výpočet ztrátové funkce (viz. \cref{sec:loss}).\\
\texttt{batteryDifferenceParameter arg} & Parametr průměrného rozdílu hladin 
baterie pro výpočet ztrátové funkce (viz. \cref{sec:loss}).\\
\texttt{waitingTimesParameter arg} & Parametr průměrné čekací doby na nabíjecích
stanicích pro výpočet ztrátové funkce (viz. \cref{sec:loss}).\\
\bottomrule
\end{tabular}
\caption{Popis přepínačů pro specifikování obecných vlastností optimalizace.}
\label{tab:prepinace_optimalizace}
\end{table}


\begin{table}
\centering\footnotesize\sf
\begin{tabular}{p{0.35\linewidth} p{0.58\linewidth}}
\toprule
Příkaz & Popis funkce \\
\midrule
\texttt{greedyMaxIterations arg} & Maximální počet iterací algoritmu. \\
\texttt{greedyNumThrowAway arg} & Maximální rozdíl počtu stanic mezi iteracemi algoritmu. \\
\bottomrule
\end{tabular}
\caption{Popis přepínačů pro specifikování parametrů hladové optimalizace (viz. \cref{sec:greedy}).}
\label{tab:prepinace_greedy}
\end{table}


\begin{table}
\centering\footnotesize\sf
\begin{tabular}{p{0.52\linewidth} p{0.41\linewidth}}
\toprule
Příkaz & Popis funkce \\
\midrule
\texttt{geneticPopulatioSize arg} & Velikost populace. \\
\texttt{geneticNumGenerations arg} & Počet generací. \\
\texttt{geneticNumBestSelection arg} & Počet nejlepších jedinců pro zkopírování
    do následující generace. \\
\texttt{geneticTournamentSelectionTreshold arg} & Pravděpodobnost výběru lepšího
    jedince v turnajové selekci. \\
\texttt{geneticMutationTreshold arg} & Pravděpodobnost mutace nabíjecí stanice. \\
\texttt{geneticMemberSizeVariance arg} & Směrodatná odchylka rozdílu velikosti 
    nového jedince a jeho rodiče pro mutaci velikosti jedince. \\
\bottomrule
\end{tabular}
\caption{Popis přepínačů pro specifikování parametrů optimalizace genetickým algorimem
    (viz. \cref{sec:genetic_optim}).}
\label{tab:prepinace_genetic}
\end{table}


\begin{table}
\centering\footnotesize\sf
\begin{tabular}{p{0.46\linewidth} p{0.47\linewidth}}
\toprule
Příkaz & Popis funkce \\
\midrule
\texttt{kMeansNumIterationsOneRun arg} & Počet iterací jednoho běhu algorimu k-means. \\
\texttt{kMeansNumGenerations arg} & Počet generací modelů. \\
\bottomrule
\end{tabular}
\caption{Popis přepínačů pro specifikování parametrů optimalizace algoritmem k-means 
    (viz. \cref{sec:kmeans_optim}).}
\label{tab:prepinace_kmeans}
\end{table}


\begin{table}
\centering\footnotesize\sf
\begin{tabular}{p{0.17\linewidth} p{0.76\linewidth}}
\toprule
Příkaz & Popis funkce \\
\midrule
\texttt{randomModels} & Spustí pevný počet simulací na modelech s náhodně 
rozmístěnými nabíjecími stanicemi. \\
\texttt{greedy} & Spustí optimalizaci hladovým algoritmem. \\
\texttt{genetic} & Spustí optimalizaci genetickým algorimem. \\
\texttt{kMeans} & Spustí optimalizaci algoritmem K-Means. \\
\bottomrule
\end{tabular}
\caption{Popis přepínačů pro volbu optimalizační metody.}
\label{tab:prepinace_volba_optim}
\end{table}


\subsection{Příklad spouštění programu}
V této sekci uvádíme příklad spuštění programu se vstupními soubory. Po řadě pro
křižovatky, silnice a města s názvy:
\begin{Verbatim}
    prepared_graph/combined_nodes.txt
    prepared_graph/edges.txt
    prepared_graph/cities.txt
\end{Verbatim}

Dále uvádíme výstupní soubory. Po řadě pro soubor s nově přidanými hranami
po sloučení komponent souvislosti, pro reprezentaci vrcholů z předpřipraveného grafu 
a pro reprezentaci hran z předpřipraveného grafu:
\begin{Verbatim}
    prepared_graph/added_edges.txt, 
    prepared_graph/prepared_nodes_new.txt
    prepared_graph/prepared_edges_new.txt
\end{Verbatim}

Dále v tomto příkladě volíme čas simulace 100 minut, celkový počet stanic 300 a
počet generací genetického algoritmu 3. Nakonec specifikujeme, že chceme spustit 
optimalizaci genetickým algoritmem a uložit předpřipravenou mapu. Další parametry
jsou zvoleny automaticky, podle výchozí hodnoty nastavené v programu, kterou lze
dohledat příkazem:

\begin{Verbatim}
    ./Traffic_Simulator --help
\end{Verbatim}

Optimalizaci na parametrech popsaných výše spustíme příkazem:

\begin{Verbatim}
./Traffic_Simulator \
--nodeCityFile="prepared_graph/combined_nodes.txt" \
--edgesFile="prepared_graph/edges.txt" \
--citiesFile="prepared_graph/cities.txt" \
--addedEdgesFile="prepared_graph/added_edges.txt" \
--preparedNodesFile="prepared_graph/prepared_combined_nodes_new.txt" \
--preparedEdgesFile="prepared_graph/prepared_edges_new.txt" \
--simulationTime=100 \
--numStations=300 \
--geneticNumGenerations=3 \
--genetic \
--savePrepared
\end{Verbatim}


Pokud zvolíme vstupní soubory (křižovatek, silnic a města):

\begin{Verbatim}
    prepared_graph/prepared_combined_nodes.txt
    prepared_graph/prepared_edges.txt
    prepared_graph/cities.txt
\end{Verbatim}

Pokud nechceme ukládat předpřipravenou podobu grafu a parametry volíme stejně jako v minulém
případě, pak příkaz pro spuštění vypadá následovně:

\begin{Verbatim}
    ./Traffic_Simulator \
    --nodeCityFile="prepared_graph/prepared_combined_nodes.txt" \
    --edgesFile="prepared_graph/prepared_edges.txt" \
    --citiesFile="prepared_graph/cities.txt" \
    --simulationTime=100 \
    --numStations=300 \
    --geneticNumGenerations=3 \
    --genetic 
\end{Verbatim}


\section{Výstup programu}
První část výstupu programu je vždy výpis použitých parametrů. Část tohoto
výpisu může vypadat například následovně:

\begin{Verbatim}
Save prepared is: 0
Edges: prepared_graph/prepared_edges.txt
Node City: prepared_graph/prepared_combined_nodes.txt
Cities: prepared_graph/cities.txt
Simulation time: 1000
Length of the edge segment: 1
Number of closest stations: 10
Car consumtion: 0.001
Number of charging stations: 500
Capacity of the charging station: 1
...
\end{Verbatim}

Následuje část kontrolních zpráv o procesu eliminace křižovatek stupně 2. 
Program vždy po eliminaci 10000 křižovatek stupně 2 vypíše
informaci o úspěšném průběhu procesu eliminace křižovatek. 
Příklad tohoto výpisu může vypadat následovně:

\begin{Verbatim}
Iteration: 0 Vertices of degree 2 merged and deleted.
Iteration: 10000 Vertices of degree 2 merged and deleted.
Iteration: 20000 Vertices of degree 2 merged and deleted.
Iteration: 30000 Vertices of degree 2 merged and deleted.
Iteration: 40000 Vertices of degree 2 merged and deleted.
Iteration: 50000 Vertices of degree 2 merged and deleted.
Iteration: 60000 Vertices of degree 2 merged and deleted.
Iteration: 70000 Vertices of degree 2 merged and deleted.
Iteration: 80000 Vertices of degree 2 merged and deleted.
Iteration: 90000 Vertices of degree 2 merged and deleted.
Iteration DONE
\end{Verbatim}


Dále následuje posloupnost kontrolních zpráv o slučování komponent 
souvislosti, která je zakončena informací o spuštění zvolené optimalizační metody. 
Tato posloupnost může vypadat následovně (pro spuštění hladové optimalizační
metody):

\begin{Verbatim}
Total number of components: 16
Total number of components: 15
Total number of components: 14
Total number of components: 13
Total number of components: 12
Total number of components: 11
Total number of components: 10
Total number of components: 9
Total number of components: 8
Total number of components: 7
Total number of components: 6
Total number of components: 5
Total number of components: 4
Total number of components: 3
Total number of components: 2
Total number of components: 1
Running greedy algorithm.
\end{Verbatim}

Dále následuje výpis kontrolních zpráv o běhu simulace po každých 
10 odsimulovaných minutách, který je zakončen výsledky simulace a informací
o hodnotě ztrátové funkce. Tento výpis se periodicky opakuje pro každou
spuštěnou simulaci. 
Níže uvádíme příklad výpisu jednoho běhu simulace pro dobu 100 minut.

\begin{Verbatim}
Time 10 elapsed!
Time 20 elapsed!
Time 30 elapsed!
Time 40 elapsed!
Time 50 elapsed!
Time 60 elapsed!
Time 70 elapsed!
Time 80 elapsed!
Time 90 elapsed!
Time 100 elapsed!
Number of stations: 500
Number of run down batteries: 654393
Average traveling duration: 98.0642
Average battery difference between start and end: -0.345312
Average waiting times in charging station: 165.868
------------------------------------------------------
Model loss: 6.54896e+07
\end{Verbatim}

Po skončení všech simulací je program zakončen finálním výpisem informací o
modelu s nejmenší hodnotou ztrátové funkce, nebo v případě volby 
\texttt{randomModels} výpisem aritmetického průměru hodnot popisující vlastnosti simulace.

Speciálním případem je optimalizace za pomoci genetického algoritmu, kde jsou
před finálním výpisem vypsány hodnoty ztrátové funkce všech modelů použitých v průběhu
optimalizace. Část finálního výpisu genetického algoritmu před výpisem informací o nejlepším
modelu může pro 3 generace a velikost populace 3 vypadat následovně:

\begin{Verbatim}
Algorithm finished
Losses for the generation: 0
Loss: 7.58681e+07
Loss: 7.5309e+07
Loss: 7.55552e+07
Losses for the generation: 1
Loss: 7.44727e+07
Loss: 7.56497e+07
Loss: 7.46101e+07
Losses for the generation: 2
Loss: 7.55472e+07
Loss: 7.50069e+07
Loss: 7.56775e+07

Total best loss is: 7.44727e+07
\end{Verbatim}

Finální výpis nejlepších výsledků všech optimalizačních metod může vypadat následovně:

\begin{Verbatim}
Best model info:
------------------------------------------------------
Number of stations: 101
Number of run down batteries: 733511
Average traveling duration: 100.09
Average battery difference between start and end: -0.34632
Average waiting times in charging station: 203.302
------------------------------------------------------
Model loss: 7.33615e+07
\end{Verbatim}

Tento výpis nám říká, že v nejlepším modelu bylo použito $101$ nabíjecích stanic,
počet vozidel, kterým došla baterie je $733511$, průměrná doba cestování vozidla
v minutách je $100.09$, průměrný rozdíl počáteční a koncové hladiny baterie je 
$-0.34632$ (záporná hodnota znamená, že vozidlo dojelo do cíle s vyšší hladinou
baterie, než s jakou vyjíždělo), průměrný čas čekání vozidel na nabíjecích stanicí
než se dostanou na řadu je $203.302$ a nakonec hodnota ztrátové funkce modelu
je $7.33615e+07$.