\chapter{Programátorská dokumentace}
\label{chap:prog_dok}

Účelem této části je především vysvětlit vzájemnou propojenost jednotlivých
tříd programu a vysvětlit jeho fungování v rámci celku. Podrobné informace o
konkrétních metodách jednotlivých tříd jsou popsány v podrobné dokumentaci
programu (viz. \cref{chap:prilohy}).


\section{Struktura programu}
Program je strukturován do vrstev tak, aby byla zajištěna jednoduchá práci s programem
na různých úrovni abstrakce. 

Uživatel pracuje pouze s nejabstraktnější vrstvou,
kterou je třída \texttt{Optimizer}, jež spravuje veškerou logiku spojenou s optimalizací.
Tato třída obaluje třídu \texttt{TimeTable}, jež je zodpovědná za správné fungování
diskrétní simulace a vytváři abstrakci pro třídu \texttt{TrafficSimulator}. Třída
\texttt{TrafficSimulator} je zodpovědná především za generování
událostí simulace (např. následující výjezd vozidla, odkud vozidlo vyjede a 
kam směřuje apod.) a propojení všech objektů simulace. Důležitým aspektem této 
vrsty je správa načítání mapy. Třída spravuje objekty třídy \texttt{MapReader}
a \texttt{GraphAdjuster} zodpovědné za správné předzpracování mapy. Dále je zodpovědná
za veškeré objekty abstraktní třídy \texttt{Vehicle}, které reprezentují vozidla v 
simulaci. V neposlední řadě také vytváří abstrakci pro třídy \texttt{Map}, která 
reprezentuje nejnižší vrstvu programu. Třída
\texttt{Map} je zodpovědná za veškeré grafové operace a veškeré operace pracující s 
reálnou zeměpisnou pozicí. Její součástí je objekt knihovny \texttt{boost}
s názvem \texttt{graph\_} reprezentující graf silniční sítě. Dále třída shlukuje 
informace o nabíjecích stanicích (reprezentované pomocí třídy \texttt{ChargingStation})
a městech náležících do simulované oblasti. 

Součástí programu je také řada dalších tříd pokrývající pomocné operace programu.
Některé z nich jsme již zmínili (\texttt{MapReader}, \texttt{GraphAdjuster}, 
\texttt{Vehicle}, 
\texttt{ChargingStation}). Za zmínku stojí ještě třídy \texttt{Car}, jež je potomkem třídy
\texttt{Vehicle} reprezentující typ vozidla "auto" (viz. \cref{subsec:druhy_vozidel}),
\texttt{Edge}, která pokrývá funkcionalitu a vlastnosti silnice v mapě (viz. \cref{def:silnice})
a \texttt{Node}, jež plní funkci křižovatek na mapě (viz. \cref{def:krizovatka}).
Zbylé třídy slouží především pro shlukování souvisejících informací 
(parametry simulátoru, optimalizací, vlastnosti nabíjecích stanic, modelu apod.).

Jedinou vyjímkou, jež není součástí struktury výše je část pracující s 
parametry programu, jež je realizována pomocí knihovny \texttt{boost}. Přesněji pomocí
funkcionalit z \texttt{program\_options}.


\section{Třída Optimizer}

Třída \texttt{Optimizer} je hlavní třídou celého programu, která na základě 
uživatelských vstupů spouští a realizuje zvolenou optimalizaci. Vytváří 
abstrakci nad třídou \texttt{TimeTable}, jež používáme pro spouštění simulací. 
Do rozhraní třídy náleží metody spouštějící optimalizační algoritmy:

\begin{verbatim}
    GreedyAlgorithm
    GeneticAlgorithm
    KMeansAlgorithm
\end{verbatim}

Metoda, jež počítá ztrátovou funkci modelu po uběhnutí simulace:
\begin{verbatim}
    ModelLoss
\end{verbatim}

A metoda, která opakovaně simuluje provoz na náhodně vygenerovaném modelu:
\begin{verbatim}
    RunMultipleSimulations
\end{verbatim}

Tato metoda slouží pro analýzu optimalizačních metod pro porovnání metod s 
náhodným přístupem. \Cref{chap:optim} nabízí podrobnější popis jednotlivých 
optimalizačních metod a ztrátové funkce. Dále třída obsahuje řadu privátních 
metod zajišťujících správnou funkčnost hlavních metod popsaných výše. Více informacím
je specifikováno v podrobné dokumentaci (viz. \cref{chap:prilohy}).

\section{Třída TimeTable}

Třída \texttt{TimeTable} je třída zodpovědná za správné fungování diskrétní
simulace. Vytváří abstrakci třídy \texttt{TrafficSimulator} a slouží jako rozhraní pro
zisk informací o vlastnostech simulace (pro analýzu optimalizačních metod). 
Do jejího rozhraní náleží především funkce spravující diskrétní simulaci.
Další skupinou metod této třídy jsou metody s předponou \texttt{Get}, 
s jejíž pomocí jsme schopni získat relevantní informace o simulaci.
Jediná metoda, jež nenáleží do zmíněných skupin je \texttt{LoadMap}. Ta slouží
jako zprostředkovatel žádosti o načtení mapy a spouští příslušné
metody ve třídě \texttt{TrafficSimulator}.

\Cref{chap:prubeh_simulace} nabízí podrobnější popis fungování diskrétní
simulace.


\section{Třída TrafficSimulator}

Třídu \texttt{TrafficSimulator} lze považovat za jádro celého simulátoru, jejíž funkce
je spravovat objekty simulace. V rámci simulace je zodpovědná především za
generování náhodných jevů (pozice nabíjecích stanic, generování vozidel, doby 
následující akce apod.) (viz. \cref{subsec:generovani_vozidel}), 
jež je zastoupeno metodami s předponou \texttt{Generate}. Pomocí této třídy jsou v simulaci
spravovány objekty vozidel (potomci třídy \texttt{Vehicle}). Zásadní vlastností této třídy
je vytváření rozhraní pro práci s třídou \texttt{Map}, jež pokrývá veškeré grafové 
operace programu. S třídou \texttt{Map} také úzce souvisí třídy 
\texttt{MapReader} a \texttt{GraphAdjuster}, sloužící pro načtení a předpřípravu
uživatelské reprezentace mapy do formátu, jenž je uložen
ve třídě \texttt{Map}. \Cref{sec:soubor_mapy} podrobně popisuje formát vstupních souborů.


\section{Třída Map}

Třída \texttt{Map} je technicky nejkomplexnější část programu. Slouží
jako rozhraní pro veškeré grafové operace a operace, jež pracují s reálnou
geografickou pozicí na mapě. Graf dopravní sítě je reprezentován pomocí 
příslušných objektů grafové části knihovny \texttt{boost}. Dále jednotlivé křižovatky 
(\cref{def:krizovatka}) a silnice (\cref{def:silnice}) jsou navíc reprezentovány 
vlastními objekty třídy \texttt{Node} (křižovatky) a \texttt{Edge} (silnice). 
Třída také sdružuje všechny nabíjecí stanice (\cref{def:nabijeci_stanice}) vyskytující 
se v mapě (reprezentovány třídou \texttt{ChargingStation}) a informace o rozmístění a 
populaci měst (\cref{def:mesto}) v simulované oblasti.

Třída obsahuje metody pro spravování podoby grafu, informací o nabíjecích 
stanicích a městech. Tyto funkcionality pokrývají metody s předponou \texttt{Add} a 
\texttt{Set}. Dále jsou zde metody pro získávání informací o objektech mapy začínající
příponou \texttt{Get} a \texttt{Exist} a metody přímo související s diskrétní simulací a 
optimalizací (\texttt{ResetSimulation} a \texttt{SimulateVehiclePassedThroughEdge}). Poslední
skupinou metod jsou nejkomplexnější operace třídy pracující s grafem, či mapou.


\subsection{Operace na mapě}

S ohledem na řešený problém je velmi důležitým aspektem návrhu hledání 
nejkratších cest. Program umí pracovat s 2 variantami algoritmu pro hledání
nejkratších cest v grafu. Těmi jsou Dijkstrův algoritmus, jenž je využívám
především pro hledání vzdáleností v grafu (např. v K-Means), a A-Star 
(\cref{sec:a_star}), jenž je používám v diskrétní simulaci pro hledání 
nejvýhodnější cesty vozidla (viz. \cref{subsec:hledani_trasy}).
Jako heuristika algoritmu A-Star je zvolena reálná geografická vzdálenost
mezi křižovatkami, která je spočítána pomocí příslušných vztahů pro výpočet
sférické vzdálenosti (viz. \cref{sect:zem_souradnice}). 

Dijkstrův algoritmus, algoritmus A-Star a výpočet sférické vzdálenosti je realizován
pomocí metod (ve stejném pořadí, jak jsou vypsány):

\begin{Verbatim}
    DijkstraFindDistances
    FindShortestPath
    ComputeSphericalDistance
\end{Verbatim}

Speciální metodou související s výpočtem nejkratší cesty v grafu je:

\begin{Verbatim}
    FindVehiclePath
\end{Verbatim}

Ta hledá nejkratší cestu mezi vrcholy, ale bere také v úvahu potřebu návštěvy nabíjecí
stanice, z níž vybere heuristicky tu nejvhodnější (viz. \cref{subsec:zajizdeni_na_nabijecku}).

Další metodou pracující s grafy je metoda pro hledání komponent souvislosti, 
využívaná v předpřípravě mapy pro simulaci.

\begin{Verbatim}
    TestComponents
\end{Verbatim}

Třída také obsahuje metodu hledající nejbližšího města od křižovatek a nejbližší
nabíjecí stanice od středů měst:

\begin{Verbatim}
    FindCityNodes
    FindCitiesNearestChargingStations
\end{Verbatim}


\section{Třídy objektů mapy}

Mezi třídy reprezentující objekty na mapě patří reprezentace křižovatek (\texttt{Node}),
silnic (\texttt{Edge}) a nabíjecích stanic (\texttt{ChargingStation}). 
Tyto třídy slouží především pro udržování informací o vlastnostech objektů,
obsahují však také jednoduchou logickou vrstvu pokrývající především správnou 
inicializaci, či aktualizaci informací v průběhu simulace. Třída \texttt{ChargingStation}
také pokrývá logickou vrstvu související s nabíjením vozidel (spravuje čekání 
vozidel v řadě, doby nabíjení apod.).


\section{Třídy pro načítání a předpřípravu dat}

Reprezentaci mapy je potřeba na začátku programu načíst z předpřipraveného 
formátu (viz. \cref{sec:soubor_mapy}) a řádně zpracovat do podoby s níž bude
simulátor dále pracovat (\cref{def:graf_site}). K tomuto
účelu slouží třídy \texttt{MapReader} a \texttt{GraphAdjuster}.

\subsection{Třída MapReader}
Třída \texttt{MapReader} převádí vstupní reprezentaci dat do požadovaného formátu
pomocí metod s předponou \texttt{Load}. Nejdůležitější metodou tohoto typu je metoda
\texttt{LoadMap}, jež načte veškeré potřebné informace a provede také předpřípravu 
mapy pro správné fungování simulátoru a zlepšení efektivity. Zmíněné úpravy
provádí s pomocí třídy \texttt{GraphAdjuster}. Dalším charakteristickou vlastností třídy
\texttt{MapReader} je možnost uložení aktuální reprezentace mapy (po předpřípravě mapy)
do souboru v požadovaném vstupním formátu pro efektivnější opakované 
načítání mapy. Mezi tyto metody patří takové, jejichž přípona je \texttt{Write}.

Třída nabízí možnost načítat a zapisovat také reprezentaci všech nabíjecích
stanic modelu metodami:

\begin{verbatim}
    LoadChargingStations
    WriteStations
\end{verbatim}

\subsection{Třída GraphAdjuster}
Vzhledem k typicky velkému rozměru vstupních dat je pravděpodobné, že výsledná
reprezentace mapy nebude splňovat veškeré potřebné vlastnosti nutné k správnému
chování programu, či bude jejich reprezentace zbytečně neefektivní. 
Typickými příklady tohoto problému je rozpad grafu na více komponent souvislosti
či vznik vrcholů stupně 2. Podrobněji tuto problematiku popisuje \cref{chap:priprava_mapy}. 

Řešení těchto potíží zajisťuje třída \texttt{GraphAdjuster} úzce pracující s
třídou \texttt{MapReader}. Ta po načtení grafu zkontroluje zda je graf souvislý
a pokud není, pak jej převede do souvislé podoby 
(viz. \cref{subsubsec:souvislost_grafu}). Dále se také zbavuje veškerých vrcholů 
stupně 2, čímž výrazně zefektivňuje budoucí simulaci (viz. \cref{subsec:problemy_modelu}). 
Operace jsou realizovány metodami:

\begin{verbatim}
    MergeTwoComponents
    MergeDegreeTwoVertices
\end{verbatim}


\section{Třída Vehicle}

Třída \texttt{Vehicle} je třída zahrnující veškeré informace o daném vozidle (\cref{sec:vozidla}),
s jejíž pomocí je možné simulovat chování vozidel a příslušně tak upravovat simulátor.
Při návrhu této třídy byla snaha co nejobecněji vyjádřit chování vozidel 
různých druhů a jednotlivé druhy vozidel simulace následně reprezentovat třídou,
jež je potomkem \texttt{Vehicle}. V naší implementaci je zvolen pouze jeden typ vozidla a 
to typ \texttt{Car}, jehož specifickou vlastností je, že po dosažení cílové 
destinace vrací zpět do místa výjezdu (viz. \cref{subsec:druhy_vozidel}).

Metody třídy \texttt{Vehicle} slouží převážně k získávání informací o vozidle
jinými objekty (např. aktuální stav baterie, aktuální pozice, 
zda míří na nabíjecí stanici apod.), případně k nastavování stavových parametrů 
vozidel (míří na nabíjecí stanici, začalo čekat v řadě na nabíjení apod.)
Objekty třídy jsou také schopny spočítat čas a hladinu baterie po průjezdu
silnicí, dokonce i posloupnosti silnic (celé cesty). Tyto informace jsou důležité
především pro diskrétní simulaci (\cref{chap:prubeh_simulace}) a při rozhodování, 
zda je potřeba jet na nabíjecí stanici (\cref{subsec:zajizdeni_na_nabijecku}).
Zbylé metody slouží pro simulaci akcí vozidel. Je zde metoda pro aktualizaci
trasy vozidla a nabití vozidla:

\begin{verbatim}
    UpdateVehiclePath
    ChargeBattery
\end{verbatim}

Nakonec obsahuje třída i metody, jež řádně simulují průjezd vozidla po silnici. 
V tomto případě jsou naimplementovány 3 varianty, jež řeší všechny možné varianty
průjezdu vozidla (\cref{subsec:udalost_presunu}) po silnici 
(průjezd silnice, výjez z vniřku ven, cesta uvnitř silnice).


\section{Pomocné třídy}

Zbylé třídy implementace slouží především k zapouzdření souvisejících informací,
jako jsou například parametry simulátoru, optimalizací, informace o pozicích na mapě, 
shrnující popis modelu a mnoho dalších. Názvy zmíněných tříd jsou:

\begin{verbatim}
    SimulationParameters
    OptimizerParameters
    MapPosition
    ModelRepresentation
\end{verbatim}

Tyto třídy slouží především k uchovávání informací o specifických objektech a
zjednodušenou manipulaci s daty.


\section{Testy programu}

K programu jsou také dodány testy zvolených problematických funkcionalit a
především pak testy načítání uživatelských dat, která jsou často i s ohledem na 
objem vstupních dat zdrojem problémů. Jedná se o testy rozhraní Microsoft Unit
Testing Framework pro C++. Návod na spuštění testů je k dohledaní v \citep{corob-msft_2022}.