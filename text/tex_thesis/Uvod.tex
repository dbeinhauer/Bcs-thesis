\chapwithtoc{Úvod}
\label{chap:uvod}

V důsledku zvyšujícího se zájmu společnosti chránit životní 
prostředí významně roste také snaha mnoha mezinárodních a vládních organizací
o nahrazení vozidel poháněných spalovácími motory na fosilní paliva za 
enviromentálně přívětivější varianty \citep{government_2022}.
V posledních letech se zejména díky výraznému technologickému 
pokroku v této oblasti jeví jako nejlepší alternativa
využití elektrických vozidel, jejichž počty v posledních letech násobně rostou 
\citep{iea_2022}. Zmíněný fenomén je ale také příčinou vyšší poptávky po nabíjecích stanicích, 
které jsou v určitých oblastech špatně dostupné a jejichž kapacita často není 
dostatečná.

Problém bohužel nemá jednoduché řešení, protože samotný proces 
naplánování rozmístění, kapacity a počtu nabíjecích stanic je značně komplexní.
Během rozmísťování stanic je vyvíjen tlak na minimalizaci počtu stanic a 
jejich kapacit, protože výstavba nové stanice je logisticky, finančně i časově
náročná. Zároveň pokud umístíme stanici do málo frekventované oblasti, pak 
potenciální užitek nabíjecí stanice nebude plně využit. Naopak nedostatečný počet 
stanic ve značně vytížených oblastech může vézt k vytváření front a nárustu čekací
doby na stanicích. To je v kombinaci s poměrně značnou časovou náročností nabíjení 
problematické a pro klienty stanic nepraktické. V neposlední řadě je také 
potřeba brát v úvahu vzdálenost a s ní související časový deficit způsobený 
cestou do stanice.

S rostoucím výpočetním výkonem se stále více nabízí řešení tohoto problému s
využitím celé škály optimalizačních technik, jako jsou například evoluční
algoritmy, či různé variace algorimů strojového učení. Zmíněné metody ovšem
často potřebují nějakou formu zpětné vazby, s jejíž pomocí ohodnocují různé
varianty řešení a volí z nich to nejoptimálnější. K tomuto účelu může dobře
posloužit vhodně navržený simulátor dopravy, jenž umožňuje analýzu různých
variant rozmístění nabíjecí stanic.


\section*{Cíl a popis práce}

Cílem této práce je navrhnout simulátor dopravy, jenž umožňuje co nejjednodušším
způsobem pracovat s různými variantami rozvržení nabíjecích stanic a ohodnocovat
jejich kvalitu. Dalším významným cílem této práce je s pomocí navrženého 
simulátoru navrhnout a analyzovat optimalizační algoritmy pro rozmístění 
nabíjecích stanic v dopravní síti.

Návrh simulátoru je založen na snadné manipulaci s různými variantami rozmístění
nabíjecích stanic. Důraz je kladen především na jednoduchou nastavitelnost 
počtu a kapacity jednotlivých stanic s ohledem na možné využití různých 
strategií pro jejich rozmísťování. V neposlední řadě je vyžadován také snadný 
přístup ke statistickým údajům relevantním k řešenému problému jako je například
průměrné vytížení stanic, čekací doba na nabití, průměrná doba cestování,
průměrná hladina nabití vozidel v daných úsecích dopravní infrastruktury a 
podobně. Na druhou stranu je samotný model provozu v jistých aspektech výrazně 
zjednodušen především z důvodu výpočetní efektivity simulátoru.

Hlavní význam simulátoru je popisovat především dálkové trasy, u kterých 
je nejčastější potřeba nabití vozidla během cesty. Zmíněné zaměření na 
dálkové cesty je motivováno předpokladem, že převážná část majitelů elektrických
vozidel má možnost své vozidlo dobít v cílové destinaci z vlastního zdroje energie 
na úroveň dostatečnou pro cesty na krátké vzdálenosti, které tak nejsou 
relevantní pro řešení našeho problému. Při návrhu je upozaděno řešení dopravních
situací na úrovni jednotlivých vozidel, jako je například projíždění křižovatek 
či tvoření dopravních zácp. Samotná simulace je realizována pomocí metody
diskrétní simulace.

K simulátoru jsou také dodány pomocné skripty pro předpřípravu volně dostupných
informací o reálné dopravní síti ze serveru Geofabric\footnote{\url{https://download.geofabrik.de/}},
s jejichž pomocí mohou být tato data převedena do vstupního formátu našeho 
simulátoru. Zmíněné skripty nejsou nutné pro správné fungování simulátoru. 
Pokud však chceme simulovat reálnou silniční síť je s ohledem na rozsáhlost dat 
téměř nerealizovatelná jejich příprava bez použití zmíněných skriptů, či 
nástroje s odpovídající funkcionalitou. 

V optimalizační části jsou analyzovány 3 optimalizační algoritmy pro nalezení
optimálního řešení. První tzv. hladový algoritmus hledá optimální řešení 
heuristicky na základě využití nabíjecích stanic v simulaci, 
další zvolené metody optimalizace je s využitím technik genetického algoritmu
a s pomocí algoritmu k-means.

Práce je rozdělena do několika kapitol. První kapitola je zaměřena na popis a definici
náročnějších pojmů a metod používaných v práci a jsou zde také zmíněny a popsány
související práce. V druhé kapitole popisujeme
proces přípravy mapy silniční sítě použité v simulátoru. Třetí kapitola popisuje
vlastnosti simulátoru. Ve čtvrté kapitole je pak popsán proces simulace dopravy.
V páté kapitole jsou popsány optimalizační metody použité v této práci. 
V šesté kapitole jsou popsány výsledky analýzy jednotlivých optimalizačních algoritmů.
V závěru práce diskutujeme výsledky práce a diskutujeme možnosti pokračování práce.
V příloze pak nalezneme uživatelskou, programátorskou dokumentaci simulátoru a
popis přílohy programu.



