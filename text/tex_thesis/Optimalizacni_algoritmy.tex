\chapter{Optimalizační algoritmy pro rozmístění nabíjecích stanic}
\label{chap:optim}

Náš simulátor poskytuje celkem 3 odlišné strategie pro optimalizaci rozmístění
nabíjecích stanic, z nichž 2 také umožňují optimalizovat počet použitých stanic.
V této kapitole zvolené algoritmy popíšeme a zadefinujeme také ztrátovou funkci
optimalizace.


\section{Ztrátová funkce (loss function)}
\label{sec:loss}

Abychom mohli optimalizovat řešení problému musíme zvolit způsob měření 
kvality jednotlivých modelů rozmístění nabíjecích stanic. K tomuto účelu slouží tzv. 
\emph{ztrátová funkce} (loss), kterou volíme podle řešeného problému.
Typicky je ztrátová funkce definována tak, aby s klesající funkční hodnotou 
rostla kvalita řešení.

Zadefinujme ztrátovou funkci, která popisuje kvalitu modelu $M$ rozmístění nabíjecích
stanic na dopravní síti simulátoru.

\begin{defn}[Ztrátová funkce modelu]\label{def:loss}
    \emph{Ztrátová funkce modelu} $M$ s parametry
    $\{p_1, p_2, p_3, p_4, p_5\} \in \mathbb{R}^+_0$, je definována jako
    $L:\mathbb{N}_0^2 \times \mathbb{R}^3 \to \mathbb{R}$, kde:
        $$L(v_1, v_2, v_3, v_4, v_5) := \sum_{i=1}^5 p_i \cdot v_i$$
    Pro hodnoty:
    \begin{itemize}
        \item $v_1 \in \mathbb{N}_0$ - počet nabíjecích stanic v modelu $M$
        \item $v_2 \in \mathbb{N}_0$ - počet vozidel, jimž se v modelu $M$ během simulace vybila baterie
        \item $v_3 \in \mathbb{R}$ - průměrné trvání cesty vozidel v minutách
        \item $v_4 \in \mathbb{R}$ - průměrný rozdíl hladin baterie vozidel na počátku a konci cesty
        \item $v_5 \in \mathbb{R}$ - průměrná doba čekání ve frontě na nabíjecí stanici v minutách
    \end{itemize}
\end{defn}


S ohledem na volbu optimalizačních algoritmů, dává smysl zaměřit se při volbě
parametrů ztrátové funkce pouze na analýzu počtu nabíjecích stanic a počtu 
vozidel, kterým se během simulace vybila baterie.
Zbylé parametry nejsme schopni s námi zvolenými optimalizačními algoritmy 
cíleně optimalizovat a mají spíše informativní funkci, pro detailnější analýzu
řešení. Z tohoto důvodu volíme pro tyto vlastnosti hodnoty parametrů,
tak aby výrazně nezasahovaly do hodnoty ztrátové funkce.

\subsection{Poznámka k vybitým vozidlům}

Je potřeba poukázat na skutečnost, že v reálném případě téměř nikdy nenastává situace,
kdy se baterie vozidla úplně vybije a vozidlo nemůže pokračovat dále v cestě.
Pokud by k tomuto jevu docházelo často, pak by bylo využití elektrických vozidel
prakticky nepoužitelné pro bězné uživatele. V našem simulátoru však k tomuto jevu
dochází nerealisticky často.

Důvodem volby návrhu takového simulátoru je jednoduchá práce s údaji o
vybitých vozidlech v optimalizačních algoritmech.
Jedná se totiž o parametr, který vždy jasně popisuje kvalitu řešení. To se 
například nedá říct o průměrném rozdílu hladin baterie na začátku a v cíli cesty, či
o průměrné době cestování vozidel.
Bylo by podstatně obtížnější odhadnout kvalitu řešení, vzhledem k těmto parametrům.

Oproti tomu počet aut s vybitou baterií dobře popisuje počet aut, které
potřebují navštívit nabíjecí stanici, která je ovšem velmi vzdálená od jejich 
trasy. Zjednodušeně řečeno, čím méně vybitých aut v simulaci, tím optimálnější je 
rozmístění stanic v modelu.

Myšlenku optimalizace řešení pomocí vybitých vozidel jsme převzali z práce autorů
\citet{kmeans_layout}.


\section{Optimalizace hladovým (greedy) algoritmem}
\label{sec:greedy}

Nejjednodušší implementovaný optimalizační algoritmus jsme pojmenovali jako tzv. 
\emph{Hladový algoritmus}.
Jedná se o heuristický postup, jehož cílem je hladově na základě 
aktuálních výsledků simulace optimalizovat pozice nabíjecích stanic a jejich počet.

Princip řešení je založen na opakovaném simulování dopravy na modelech, ve 
kterých jsou nabíjecí stanice rozmísťovány pomocí jednoduché heuristiky 
založené na výsledcích předchozí simulace. K jeho popisu se hodí zadefinovat
pojem tzv. \emph{užitečné nabíjecí stanice}.

\begin{defn}[Užitečnost nabíjecí stanice]\label{def:uzitecna_stanice}
    Nabíjecí stanici nazveme \emph{užitečnou}, pokud byla v nějakém okamžiku
    simulace použita alespoň jedním vozidlem. V opačném případě nazveme stanici 
    \emph{zbytečnou}.
\end{defn}

Dále pracujeme s parametrem $r \in \mathbb{N}_0$, pro který musí platit
$|M_0| - |M_1| \leq r$, kde $|M|$ značí počet nabíjecích stanic v modelu $M$.
Parametr $r$ tedy popisuje, o kolik se může zmenšit celkový počet nabíjecích 
stanic v modelu $M_1$ oproti modelu $M_0$.

Algoritmus nejprve spustí simulaci na aktuálním modelu $M_0$. Následně nalezne
všechny užitečné nabíjecí stanice $U_0$ a vloží je do novém modelu $M_1$. Zbytečné 
nabíjecí stanice v novém modelu nepoužívá. Po nalezení všech užitečných stanic 
spočítá rozdíl $r_0 = |M_0| - |U_0|$. Pokud $r_0 \leq r$, pak nabíjecí stanice
modelu $M_1$ jsou právě všechny užitečné stanice. Pokud je rozdíl počtu stanic 
vyšší než $r$, pak do $M_1$ náhodně dogenerujeme stanice, aby $|M_0| - |M_1| = r$.

Tento postup opakujeme iterativně do doby než je buď dosaženo maximálního
povoleného počtu iterací algoritmu, nebo pokud je rozdíl mezi hodnotami chybových funkcí dvou posledních 
modelů menší než zvolená hranice (algoritmus dokonvergoval k řešení).

\begin{algorithm}
\begin{algorithmic}
\Function{Hladová optimalizace}{\newline $s_0$ - počáteční počet stanic,\newline
    $r$ - nejvyšší povolený rozdíl mezi iteracemi,\newline
    $iter_{max}$ - maximální počet iterací,\newline
    $loss_{diff}$ - hranice konvergence (rozdíl chybové funkce mezi iteracemi)\newline
    }
    \State Náhodně inicializuj model $M$ s celkovým počtem stanic $s_0$.
    \State $iter \gets 0$   \Comment Aktuální číslo iterace.
    \State $loss_{prev} \gets 0$    \Comment Hodnota chybové funkce minulého modelu.
    \While{$iter < iter_{max}$}
        \State Simuluj dopravu na modelu $M$.
        \State $loss_{curr} \gets$ hodnota chybové funkce modelu $M$
        \If{$|loss_{prev} - loss_{curr}| < loss_{diff}$}
            \If{$Iter > 0$} \Comment Přeskoč pokud se jedná o první iteraci.
                \State Ukonči optimalizaci.
            \EndIf
        \EndIf
        \State $s \gets$ počet stanic modelu $M$
        \State $U \gets$ užitečné stanice modelu $M$
        \If{$s - |U| \leq r$}
            \State $M \gets$ model s nabíjecími stanicemi $U$
        \Else
            \State $r_{curr} \gets s - |U|$ \Comment Počet stanic pro náhodné vygenerování.
            \State $S_{rand} \gets$ Náhodně vygeneruj $r_{curr}$ stanic.
            \State $M \gets$ model s nabíjecími stanicemi $U$ a $S_{rand}$
            \State $loss_{prev} \gets loss_{curr}$
        \EndIf
        \State $iter \gets iter + 1$
    \EndWhile
\EndFunction
\end{algorithmic}
\caption{Algoritmus optimalizující pozice a počet nabíjecích stanic hladovým způsobem.}
\label{alg:greedy}
\end{algorithm}


\section{Optimalizace genetickým algoritmem}
\label{sec:genetic_optim}

Další optimalizační metodou je optimalizace genetickým algoritmem 
(viz. \cref{section:genetic_alg}). Použití genetického algoritmu pro zvolený
optimalizační problém je inspirováno prací autorů \citet{niccolai2021optimization}.
Metoda umožňuje podobně jako optimalizace hladovým algoritmem optimalizovat
také počet nabíjecích stanic.

Jedinci v optimalizaci reprezentují jedno rozmístění nabíjecích stanic.
Pro snadnou manipulaci kódujeme jedince jako uspořádanou n-tici nabíjecích
stanic (kde $n$ značí celkový počet stanic). U jedinců nepožadujeme stejnou 
velikost a je tak možné optimalizovat i počet nabíjecích stanic.
Jako fitness jedince volíme loss modelu po simulaci (\cref{def:loss}).
Musíme brát v potaz, že v našem případě mají lepší jedinci nižší hodnotu fitness,
oproti typické variantě GA, kde vyšší fitness typicky značí lepší kvalitu řešení.

Pracujeme s variantou GA, kde je počet jedinců v různých generacích stejný.
Optimalizace GA pak probíhá následovně. Nejprve zkopírujeme zvolený počet nejlepších
jedinců aktuální generace do nové generace. Zbylé jedince dostaneme křížením.
Individua pro křížení volíme turnajovou selekcí. Dále aplikujeme jednobodové křížení,
v němž využijeme zakódování jedince jako uspořádanou n-tici stanic.
Náhodně zvolíme bod křížení jako pozici v reprezentaci jedince s 
menším počtem stanic. Reprezentaci jedince rozdělíme pomocí tohoto bodu na dvě části 
(první část je před zvoleným bodem, druhá je za ním).
Následně mezi jedinci určenými pro křížení vyměníme druhou část reprezentace a
dostaneme tak jedince nové generace. 

Nové jedince ještě následně mutujeme. Mutaci dělíme na dvě podčásti.
V první části procházíme každou nabíjecí stanici jedince a se zvolenou (malou)
pravděpodobností ji nahradíme novou náhodně vygenerovanou. 
Ve druhé části náhodně odebíráme, či přidáváme nové náhodně vygenerované stanice.
Tato část mutace nám umožňuje získávat jedince odlišných velikostí 
(tedy i zkoumat varianty řešení s proměnlivým počtem nabíjecích stanic).


\begin{algorithm}
\begin{algorithmic}
\Function{Optimalizace GA}{\newline $num_{gen}$ - počet generací,\newline
    $pop_{size}$ - velikost populace,\newline 
    $k$ - počet nejlepších jedinců pro přímou volbu do nové populace \newline
    }
    \State Nahodně inicializuj první generaci jedinců a na každém spusť simulaci.
    \State $iter \gets 0$   \Comment Aktuální číslo iterace.
    \While{$iter < num_{gen}$}
        \State Zkopíruj $k$ nejlepších jedinců do nové populace.
        \State $pairs \gets$ Turnajovou selekcí zvol $(pop_{size} - k) / 2$ dvojic pro křížení.
        \ForAll{pairs}
            \State Proveď jednobodové křížení na dvojici jedinců.
            \State Proveď mutaci na zkřížených jedincích.
            \State Přidej zmutované jedince do nové populace.
        \EndFor
    \EndWhile
\EndFunction
\end{algorithmic}
\caption{Algoritmus optimalizující pozice a počet nabíjecích stanic s pomocí genetického algorimu.}
\label{alg:genetic}
\end{algorithm}


\section{Optimalizace inspirovaná k-means}
\label{sec:kmeans_optim}

Poslední optimalizace použita v práci je inspirována algoritmem k-means 
(viz. \cref{sec:kmeans_alg}). Návrh algoritmu byl také inspirován optimalizačním
algoritmem z \citet{kmeans_layout}. Jako jediná z optimalizačních metod optimalizuje
tato metoda pouze pozice nabíjecích stanic.

Algoritmus je založen na myšlence, že centroidy z algoritmu k-means jsou pozice
nabíjecích stanic. Zvolené $k$ z algoritmu tedy udává počet nabíjecích stanic.
V prvním kroce algoritmu jsou centroidy generovány náhodně.
Následně probíhá krok podobný hledání shluků. Pro každou křižovatku v grafu 
(viz. \cref{def:krizovatka}) spočítáme vzdálenost od každého centroidu pomocí
Dijkstrova algoritmu. Křižovatka pak náleží shluku, který odpovídá nejbližšímu
centroidu od křižovatky.

Komplikovanější část algoritmu je metoda přepočítávání nové pozice centroidu. 
Předpokladem pro správnou funkci je přístup k informacím o průměrné hladině
baterie vozidel ve vrcholech grafu (viz. \cref{def:segment_silnice}). 

Algoritmus nejprve spočítá tzv. \emph{váhu křižovatky}. Ta popisuje "míru vybitosti"
vozidel v okolí křižovatky. Při výpočtu váhy libovolné křižovatky 
$k$ procházíme všechny silnice $r$ napojeny na křižovatku $k$. 
Každá silnice poskytuje také informaci o průměrné hladině nabití vozidel,
jež silnicí projížděly. Informace o průměrné hladině nabití vozidel je uložena
v segmentech silnice. Každý segment $s$ silnice $r$ přispěje k váze křižovatky 
$k$ hodnotou závislou na vzdálenosti segmentu od křižovatky $v$ 
a hodnotě $s.b$ (průměrné hladině baterie v segmentu $s$). Takto projdeme 
všechny křižovatky grafu a všechny silnice, které jim náleží a dostaneme tak váhu
pro každou křižovatku grafu. 

Na váhu křižovatky se dá z pohledu algoritmu k-means nahlížet tak, že váha 
udává počet bodů v prostoru, jež jsou umístěny na pozici křižovatky $k$. 
Hlavní motivací pro zadefinování této veličiny je vytvoření vhodné abstrakce
pro aplikaci algoritmu k-means.


\begin{algorithm}
\begin{algorithmic}
\Function{Výpočet váhy křižovatky}{\newline$k$ - křižovatky, jejíž váhu chceme spočítat}
    \State $w \gets 0$ \Comment Počáteční váha křižovatky $k$ je 0.
    \ForAll{silnice $r$ náležící vrcholu $k$}
        \State $w_r \gets 0$ \Comment Váha silnice $r$.
        \State $n \gets$ |r.S| \Comment Počet segmentů silnice.
        \ForAll{segment $s$ silnice $r$}
            \State $i \gets$ vzdálenost segmentu $s$ od křižovatky $k$ \Comment V počtech segmentů.
            \State $avg \gets$ průměrná hladina baterie v segmentu $s$ \Comment V rozmezí (0-1).
            \State $w_r \gets w_r + 1 - i / n$ \Comment Přičti k váze silnice $r$, váhu segmentu $s$.
        \EndFor
        \State $w \gets w + w_r$ \Comment Přičti k váze křižovatky $k$ váhu silnice $r$.
    \EndFor
    \State \Return $w$

\EndFunction
\end{algorithmic}
\caption{Funkce pro výpočet váhy křižovatky.}
\label{alg:vert_weight}
\end{algorithm}


Po spočítání vah křižovatek je aplikován klasický postup aktualizace nové pozice 
centroidu algoritmu k-means. Spočítá se vážený průměr souřadnic všech křižovatek 
náležících do zvoleného shluku s pomocí vah křižovatek. Souřadnice z nichž 
počítáme průměr jsou reálné zeměpisné souřadnice (viz. \cref{sect:zem_souradnice}) 
křižovatek. 

Vypočtená průměrná pozice $P$ ale téměř nikdy nebude odpovídat pozici nějaké křižovatky
v grafu, jež jsou jediné objekty v grafu simulace, u nichž známe zeměpisné souřadnice.
Pozice $P$ tak nemůže být novým centroidem, jelikož náš optimalizační algoritmus
přiřazuje vrcholy do shluků na základě vzdáleností v grafu. Je tedy potřeba 
spočtenou pozici aproximovat s nějakou pozicí v grafu.

Vzhledem k paměťovým a výpočetním omezením simulátoru jsme si schopni pamatovat
pouze informace o zeměpisných souřadnicích křižovatek. Musíme tedy vypočtenou pozici 
řádně aproximovat pouze za pomoci souřadnic křižovatek. Mějme tedy centroid $centr$.
Nejprve nalezneme nejbližší křižovatku $k$ od $centr$ (ze zeměpisných souřadnic).
Následně nalezneme nejbližší sousední křižovatku $k_{neigh}$ křižovatky $k$ od
hledaného centroidu $centr$. Tímto postupem odhadujeme, že nová pozice centroidu 
leží na silnici $r$ mezi křižovatkami $k$ a $k_{neigh}$. Zároveň protože 
je křižovatka $k$ blíže hledané pozici než $k_{neigh}$, předpokládáme, že
hledaná pozice bude náležet segmentu $s$ silnice $r$, jež je blíže ke křižovatce
$k$ než ke křižovatce $k_{neigh}$. Segment $s$, jenž je vhodnou aproximací 
hledané pozice, volíme uniformně náhodně ze všech segmentů z $r.S$, které
jsou blíže vrcholu $v$. Takto dostaneme novou pozici centroidu a v některých
případech také nabíjecí stanice, která je dána segmentem $s$.

\begin{algorithm}
\begin{algorithmic}
\Function{Nejbližší pozice grafu od zeměpisné pozice}{\newline
    $coord$ - zeměpisné souřadnice hledané pozice}
    \State $k \gets$ Nejbližší křižovatka od pozice se souřadnicemi $coord$.
    \State $k_{neigh} \gets$ Soused křižovatky $k$, který je nejblíže od pozice $coord$.
    \State $r \gets$ - Silnice dána křižovatkami $k$ a $k_{neigh}$.
    \State $best_{pos} \gets$ Uniformně náhodně vygenerovaný segment silnice $r$,
    který je blíže od křižovatky $k$.
\EndFunction
\end{algorithmic}
\caption{Funkce pro nalezení nejbližší pozice v grafu od zeměpisné souřadnice.}
\label{alg:find_closest_position}
\end{algorithm}

Po nalezení nového centroidu následně nalezneme nové shluky a postup výše 
opakujeme ve zvoleném počtu iterací. Po uplynutí všech iterací vytvoříme nový model
v němž budou nabíjecí stanice umístěny na pozicích centroidů. Spustíme simulaci
na novém modelu a následně znovu aplikujeme postup podobný algoritmu k-means.
Tento postup taktéž opakujeme zvolený počet iterací.

Postup hledání nejbližšího pozice v grafu od geografické souřadnice je pouze
aproximační a nezaručuje nalezení reálně nejbližšího bodu. Předpokládáme však, 
že tento postup odhadne pozici nejbližšího bodu s dostačující přesností.

\begin{algorithm}
\begin{algorithmic}
\Function{Optimalizace pomocí k-means}{\newline
    $means_{iter}$ - počet iterací cyklu algoritmu k-means,\newline
    $sim_{iter}$ - počet simulací modelu (počet spuštěných algoritmů k-means)\newline
    }
    \State Náhodně vygeneruj centroidy.
    \State $iter_1 \gets 0$   \Comment Aktuální číslo iterace. 
    \While{$iter_1 < sim_{iter}$}
        \State Vytvoř model s nabíjecími stanicemi na pozicích centroidů.
        \State Spusť simulaci.
        \State Spočítej váhy vrcholů.
        \State $iter_2 \gets 0$
        \While{$iter_2 < means_{iter}$}
            \State Nalezni shluky centroidů.
            \State Nalezni optimální geografické pozice nových centroidů.
            \State Aproximačně nalezni pozice nových centroidů.
        \EndWhile
    \EndWhile
\EndFunction
\end{algorithmic}
\caption{Algoritmus optimalizující pozice nabíjecích stanic s pomocí algorimu k-means.}
\label{alg:kmeans}
\end{algorithm}